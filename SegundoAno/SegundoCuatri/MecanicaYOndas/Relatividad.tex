%! TEX root = Mecanica.tex

\documentclass{./Mecanica.tex}

\begin{document}
\chapter{Relatividad especial}
Supongamos un objeto sometido a una fuerza constante $F$. La aceleración de este objeto es $a=g=9.8 \frac{m}{s^{2}}$. Si partimos del reposo, la velocidad de este objeto cuando pase un día es $v \approx 8.5 \cdot 10^{5} \frac{m}{s}$. En un año, esta velocidad pasa a ser $v \approx 3.1 \cdot 10^{8} \frac{m}{s}$. Sin embargo, en la realidad, la partícula se acercaría asintóticamente a la velocidad de la luz, sin llegar a alcanzarla.\\
Por otro lado, introduzcamos un electrón en un acelerador de partículas. Este se somete a $\Delta V = 100~(V)$. El electrón acabará obteniendo una velocidad $v\approx 6\cdot 10^{6}~\frac{m}{s}$. Esto es teóricamente posible en un condensador de unos 5 milímetros. Si en su lugar $\Delta V = 13MeV$. Con esto, su velocidad sería $v\approx 2.3\cdot 10^{9} \frac{m}{s}$.  
\end{document}
