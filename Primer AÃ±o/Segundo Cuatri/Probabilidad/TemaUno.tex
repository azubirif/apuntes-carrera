%! TEX root = Probabilidad.tex

\documentclass{./Probabilidad.tex}

\begin{document}
\chapter{Nociones Básicas}
\section{Experimentos Aleatorios}
\subsection{Espacio muestral}
Son sucesos aleatorios repetidos cuyos resultados no se pueden determinar de
antemano. El objetivo es describir el fenónemo desde un punto de vista aleatorio
que describa el proceso.\\
El espacio muestral es el conjunto formado por todos los posibles resultados:
$$
\Omega = \{ \omega_{1},\dots ,\omega_{n} \}
$$
Y tenemos diferentes tipos:
\begin{itemize}
	\item Discreto: finito o numerable.
	\item Continuo.
\end{itemize}
Un suceso es cualquier subconjunto del espacio muestral.
El proceso de codificación es el que consiste en pasar de una variable cualitativa a una cuantitativa.
\subsection{Conjunto de partes}
El conjunto de partes del espacio muestral es el conjunto de todos los posibles resultados de un experimento.
\section{Probabilidad}
Cualesquiera que sean $A$, se tiene que $0\leq n(A)\leq n$, y que por tanto
$$
\lim_{ n \to \infty } \frac{n(A)}{n}=P(A)
$$
Si $A$ ocurre siempre, $n(A)=n \implies P(A)=1$
Además, si $A$ y $B$ son excluyentes, se tiene que $P(A \cap B) = 0$, y que $P(A \cup B) = P(A)+P(B)$.
\section{Axiomas de Kolmogorov}
Las propiedades anteriores dotan a $\mathbb{A}, \cap, \cup$ de una estructura de álgebra de Boole ($\sigma$-álgebra).
Una probabilidad $P$ definida sobre un álgebra de sucesos $A$, de un espacio muestral finito $\Omega$, es una función $P:A\to[0,1]$.
\begin{itemize}
	\item $P(\Omega)=1$
	\item $P(A) \in [0,1] \forall A$
	\item Si $A$ y $B$ son excluyentes, se tiene que $P(A \cap B) = 0$, y que $P(A \cup B) = P(A)+P(B)$.
\end{itemize}
También destacamos las leyes de De Morgan:
\begin{equation}
	\begin{split}
		\overline{A \cup B} &= \bar{A} \cap \bar{B}\\
		\overline{A \cap B} &= \bar{A} \cup \bar{B}
	\end{split}
\end{equation}
\subsection{Propiedades}
\begin{itemize}
	\item $P(\phi)=0$
	\item $P(\bar{A})=1-P(A)$
	\item $P(A-B)=P(A)-P(A \cap B)$, y además si $B \subset A$, $P(A-B)=P(A)-P(B)$
	\item $P(A \cup B)=P(A)+P(B)-P(A \cap B)$
	\item $P(A_{1}\cup A_{2}\cup A_{3})=\sum_{i=1}^3P(A_{i})-\sum_{i<j}P(A_{i}\cap A_{j})+P(A_{1}\cap A_{2}\cap A_{3})$
\end{itemize}
\end{document}
