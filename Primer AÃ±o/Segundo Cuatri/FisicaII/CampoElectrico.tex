%! TEX root = FisicaII.tex

\documentclass{../FisicaII.tex}

\begin{document}
\chapter{Campo eléctrico}
\begin{defin}
	Definimos la \textbf{fuerza de Coulomb} entre dos cargas, siendo una fija
	(foco) y otra la que se mueve (receptora).
	\begin{equation}
		\begin{split}
			\vec{F} = \frac{1}{4\pi \varepsilon_{0}} \frac{q_1q_2}{r^{2}}\vec{u}_{r}
			= k_{0} \frac{q_1q_2}{r^{2}}\vec{u}_{r}~(N)
		\end{split}
	\end{equation}
	Siendo $k_{0} = 9 \cdot 10^{9}~( \frac{Nm^{2}}{r^{2}})$, $\varepsilon_{0}$
	la permitividad eléctrica del vacío, y $\vec{u}_{r}$ el vector que va desde
	el foco al receptor.
\end{defin}
\begin{defin}
	El campo eléctrco es una region en el espacio alrededor de una carga en la que
	se pueda situar otra carga receptora para que esta sufra una fuerza
	repulsora o atractora.
	\begin{equation}
		\begin{split}
			\vec{E} = \frac{1}{4\pi \varepsilon_{0}} \frac{q}{r^{2}}\vec{u}_{r}
			~\left(\frac{N}{C}\right)
		\end{split}
	\end{equation}
	Podemos representar un campo eléctrico mediante punto-vector o líneas de
	fuerza.
\end{defin}
\begin{figure}[ht]
    \centering
    \incfig{fuerza-coulumb}
    \caption{fuerza-coulumb}
    \label{fig:fuerza-coulumb}
\end{figure}
\begin{defin}
	Las líneas de fuerza es aquella que tienen como origen el foco, y su vector
	tangente es el campo eléctrico en cada uno de sus puntos.
\end{defin}
\subsection{Sistema de cargas puntuales}
Supongamos un conjunto de $N$  cargas $q_{1},\dots ,q_{n}$ y están situadas en
sus vectores $\vec{r}_{1},\dots ,\vec{r}_{n}$. La fuerza que actúa sobre una
carga receptora $Q$ cuyo vector posición es $\vec{r}_{p}$ es la suma vectorial
de las fuerzas creadas por el resto de las cargas:
\begin{equation}
	\begin{split}
		\vec{F}_{p} &= \sum_{i=1}^{N} \frac{1}{4\pi\varepsilon_{0}}
		\frac{q_{i}Q}{|r_{p}-r_{i}|^{3}}(r_{p}-r_{i}) ~(N)\\
					&= \frac{Q}{4\pi\varepsilon_{0}}\sum_{i=1}^{N}
					\frac{q_{i}}{|r_{p}-r_{i}|^{3}}(r_{p}-r_{i})~(N)\\
					&= Q \sum_{i=1}^{N} \vec{E}_{i}~(N)
	\end{split}
\end{equation}
\section{Sistemas de coordenadas}
Para unas coordenadas necesitamos un origen $\vec{0}$, y una base. Para unas coordenadas cartesianas, tenemos los vectores $\vec{i}$ y $\vec{j}$. Sin embargo, también podemos utilizar las coordenadas polares, que consisten en una distancia al origen $r$, y un ángulo respecto al eje horizontal $\phi$.\\
Para definir el ángulo, hay que tener en cuenta que, al desplazar el vector un ángulo, proyectamos el vector sobre el eje $x$ positivo, y giramos el vector en sentido anti horario hasta alcanzar la posición del vector original. Ese ángulo es nuestra coordenada $\phi$.
Si trazamos el vector unitario de $r$, su vector perpendicular $\vec{u}_{\phi}$ es tangencial al ángulo $\phi$.
Para poder cambiar entre coordenadas cartesianas y polares, utilizamos las siguientes ecuaciones:
\begin{equation}
	\begin{split}
		x &= |r|\cos \phi\\ y &= |r|\sin \phi\\r^{2}&= x^{2}+y^{2} \\
\phi &= \arctan \frac{y}{x}\\
\vec{u}_{r} &= (\cos \phi, \sin \phi) \\
\vec{u}_{\phi} &= (-\sin \phi, \cos \phi)
	\end{split}
\end{equation}
Respecto a la velocidad en coordenadas polares, tenemos el siguiente desarrollo:
\begin{equation}
	\begin{split}
		v^{2}&=\dot{x}^{2}+\dot{y}^{2} \\
&=\dot{r}^{2}\cos ^{2}\phi-2\dot{r} r \dot{\phi}\sin \phi \cos \phi+r^{2}\sin ^{2}\phi
\dot{\phi}^{2} \\
&+\dot{r}^{2}\sin ^{2}\phi+2 \dot{r}r \dot{\phi}\sin \phi \cos \phi+ r^{2}\cos ^{2}\phi \dot{\phi}^{2} \\
&= \dot{r}^{2}+r^{2} \dot{\phi}^{2}
	\end{split}
\end{equation}
\section{Distribuciones de cargas continuas}
En estas distribuciones se considera que no se pueden distinguir dos cargas vecinas. Por tanto, consideramos que la carga total $Q$ se distribuye de forma continua a lo largo de un determinado volumen, superfície, etc. Esta aproximación es válida cuando se estudia el campo eléctrico a grandes distancias del objeto cargado.\\
Vamos a estudiar las diferentes distribuciones:
\begin{itemize}
	\item Lineal: carga distribuida a lo largo de un camino:
		\[
			\lambda=\frac{Q}{l}
		\]
	Cuando este valor es constante, la densidad es \textbf{homogénea}. Si tomamos un incremento de longitud $dl$, encontramos un diferencial de carga $dq$, obteniendo que
	\[
		\lambda = \frac{dq}{dl}
	\]
\item Superficial: carga distribuida a lo largo de una superfície:
	\[
		\lambda = \frac{dq}{dl}
	\]
\item Volumétrica: carga en un volumen:
	\[
		\sigma = \frac{dq}{dS}
	\]
\end{itemize}
\section{Flujo eléctrico}
El flujo eléctrico es una medida del flujo del campo eléctrico que atraviesa una superfície:
$$
\phi = \int_{S} \mathbf{E}\cdot \mathbf{n}~dA~\left( \frac{N\cdot m^{2}}{C} \right)
$$
Donde $\mathbf{E}$ es el campo eléctrico y $\mathbf{n}$ es el vector normal a la superfície en cada punto.
Si dividimos la superficie total en áreas infinitesimalmente pequeñas $dA$. El flujo infinitesimal $d\phi$ que atraviesa esa superficie es
$$
d\phi=\mathbf{E}\cdot \mathbf{n}~dA
$$
El flujo que atraviesa una superficie cerrada se expresa de la siguiente forma:
$$
\phi = \oint_{S} \mathbf{E}\cdot d \mathbf{s}
$$
\begin{teorema}
El teorema establece que el flujo neto que atraviesa una superficie cerrada $A$ es igual a la carga total $Q$ entre la permitividad del espacio encerrada por la superficie:
$$
\phi_{neto}=\oint_{A} \mathbf{E}\cdot \mathbf{n}~dA=\frac{Q_{interior}}{\varepsilon_{0}}
$$
\end{teorema}
\section{Coordenadas cilíndricas}
Las coordenadas cilíndricas se definen como
$$
\mathbf{r}=(\rho, \phi, z)
$$
Donde $\rho$ es el radio en el plano $XY$, $\phi$ es el ángulo (en sentido antihorario) con respecto al eje $X$, y $z$ es la "altura" o desplazamiento respecto al eje $XY$. En coordenadas cartesianas, esto sería
$$
\mathbf{r}=(\rho \cos \phi, \rho \sin \phi, z)
$$
La base definida por este sistema de coordenadas es:
$$
\{(\cos \phi,\sin \phi,0), (-\sin \phi,\cos \phi), (0,0,1) \}
$$
\section{Coordenadas esféricas}
El punto $P$ se define mediante unas coordenadas esféricas.
$$
P = (r,\phi,\theta):~\theta \in[0,\pi],~\phi \in[0,2\pi]
$$
En coordenadas cartesianas, esto pasa a ser:
$$
P=(r\sin\theta \cos \phi, r\sin\theta \sin \phi, r\cos\theta)
$$
\section{Electrización de un cuerpo}
\begin{defin}[Dipolo eléctrico]
Un sistema formado por un centro de carga (positivo y negativo). La carga positiva es igual y opuesta a la negativa. La distancia entre ambos centros son constantes.
\end{defin}
\begin{defin}[Momento dipolar]
A todo dipolo se le asocia un momento dipolar $\mathbf{p}$. El origen de este vector está en el centro de carga negativo. La dirección de este es la recta que une el centro de carga negativo con el positivo.\\
El módulo de este viene dado por
\[
	|\mathbf{p}|=q\cdot d~(C\cdot m)=(e\cdot pm)
\]
Cada enlace tendrá asociado un momento dipolar, y el total será la suma de todos ellos. Una molécula es apolar si su momento dipolar total es $0$.
\end{defin}
\subsection{Campo eléctrico por un dipolo}
\begin{equation}
	\begin{split}
		\mathbf{E}_{r}&=\frac{2p\cos\theta}{4\pi\varepsilon_{0}r^{3}} \mathbf{u}_{r}\\
\mathbf{E}_{\theta}&= \frac{p \sin\theta}{4\pi\varepsilon_{0}r^{3}}\mathbf{u}_{\theta}
	\end{split}
\end{equation}
\subsection{Par de fuerza sobre un dipolo en un campo exterior}
Situamos un dipolo en el interior de un campo eléctrico dado por $\mathbf{E}=E_{0}\mathbf{u}_{r}$. El dipolo gira, y se genera un momento de fuerza dado por
$$
\vec{\tau}= \vec{p}\times \vec{E}
$$
\end{document}
