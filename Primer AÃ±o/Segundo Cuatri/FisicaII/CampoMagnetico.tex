%! TEX root = FisicaII.tex

\documentclass{./FisicaII.tex}

\begin{document}
\chapter{Campo magnético}
\begin{defin}
    Un campo magnético es una región alrededor de una carga en movimiento que es capaz
de atraer o repeler otra carga en movimiento.
\end{defin}
Las líneas de campo asociadas a una carga en movimiento siempre son concéntricas
alrededor de la carga, y son perpendiculares al vector desplazamiento.
\begin{figure}[h]
    \centering


\tikzset{every picture/.style={line width=0.75pt}} %set default line width to 0.75pt        

\begin{tikzpicture}[x=0.75pt,y=0.75pt,yscale=-1,xscale=1]
%uncomment if require: \path (0,300); %set diagram left start at 0, and has height of 300

%Straight Lines [id:da4108085125035368] 
\draw    (321,50.7) -- (321,224) ;
\draw [shift={(321,48.7)}, rotate = 90] [color={rgb, 255:red, 0; green, 0; blue, 0 }  ][line width=0.75]    (10.93,-3.29) .. controls (6.95,-1.4) and (3.31,-0.3) .. (0,0) .. controls (3.31,0.3) and (6.95,1.4) .. (10.93,3.29)   ;
%Shape: Ellipse [id:dp6825538152885552] 
\draw  [dash pattern={on 4.5pt off 4.5pt}] (238,154.85) .. controls (238,135.44) and (274.99,119.7) .. (320.62,119.7) .. controls (366.24,119.7) and (403.23,135.44) .. (403.23,154.85) .. controls (403.23,174.26) and (366.24,190) .. (320.62,190) .. controls (274.99,190) and (238,174.26) .. (238,154.85) -- cycle ;
%Shape: Circle [id:dp3189869673475786] 
\draw  [fill={rgb, 255:red, 0; green, 0; blue, 0 }  ,fill opacity=1 ] (314.92,154.85) .. controls (314.92,151.7) and (317.47,149.15) .. (320.62,149.15) .. controls (323.76,149.15) and (326.32,151.7) .. (326.32,154.85) .. controls (326.32,158) and (323.76,160.55) .. (320.62,160.55) .. controls (317.47,160.55) and (314.92,158) .. (314.92,154.85) -- cycle ;
%Straight Lines [id:da6254245189497872] 
\draw    (403.23,128.7) -- (403.23,154.85) ;
\draw [shift={(403.23,126.7)}, rotate = 90] [color={rgb, 255:red, 0; green, 0; blue, 0 }  ][line width=0.75]    (10.93,-3.29) .. controls (6.95,-1.4) and (3.31,-0.3) .. (0,0) .. controls (3.31,0.3) and (6.95,1.4) .. (10.93,3.29)   ;

% Text Node
\draw (335,145.4) node [anchor=north west][inner sep=0.75pt]    {$q$};
% Text Node
\draw (330,60.4) node [anchor=north west][inner sep=0.75pt]    {$\vec{I}$};
% Text Node
\draw (425,138.4) node [anchor=north west][inner sep=0.75pt]    {$\vec{B}$};


\end{tikzpicture}
\caption{Campo magnético por carga}
\end{figure}
\section{Fuerza magnética sobre una carga en movimiento}
Supongamos una carga en movimiento \(q\) con velocidad \(\vb{v}\), que se desplaza
en el interior de un campo magnético uniforme \(\vb{B}\). La fuerza magnética sobre
dicha carga está dada por
\begin{equation}
    \begin{split}
        \vb{F} = q (\vb{v} \times \vb{B})
    \end{split}
\end{equation}
y
\begin{equation}
    \begin{split}
        |\vb{F}| = |q| |\vb{v}| |\vb{B}| \sin \theta 
    \end{split}
\end{equation}
La dirección de la fuerza es la normal del plano generado por \(\vb{v}\) y \(\vb{B}\).
El sentido de la fuerza se puede determinar haciendo como que recorremos el ángulo
que va desde \(\vb{v}\) hasta \(\vb{B}\) (el más pequeño). Suponiendo \(q > 0\):
\begin{itemize}
    \item Si es antihorario, es saliente.
    \item Si es horario, es entrante.
\end{itemize}
\section{Ley de Laplace}
Define la fuerza magnética sobre un elemento de carga que se desplaza por un hilo
conductor. Supongamos un hilo conductor de sección \(S\) y de longitud \(L\),
en el que circula una corriente \(I\). Este está situado en el interior de una región
en la que existe un campo magnético uniforme. En un intervalo de tiempo limitado por
\(t_{0}\) y \(t+ \dd{t}\), un elemento de carga \( \dd{q}\) se desplaza una distancia
\( \dd{\vb{l}}\). Las cargas se mueven con velocidad uniforme \(\vb{v}\). La fuerza
magnética que actua sobre \( \dd{q}\) es igual a
\begin{equation}
    \begin{split}
        \dd{\vb{F}} = \dd{q}(\vb{v} \times \vb{B})
    \end{split}
\end{equation}
teniendo
\[
    \dd{\vb{F}} = \dv{q}{t} \dd{t} (\vb{v} \times \vb{B}) =
    I ( \vb{v} \dd{t} \times \vb{B}) = I ( \dd{\vb{l}} \times \vb{B})
\]
y por tanto
\[
    \int \dd{\vb{F}} = \int_{0}^L I ( \dd{\vb{l}} \times \vb{B})
\]
Y si "todo" es constante, tenemos
\begin{equation}
    \begin{split}
        \vb{F} = I(\vb{L} \times \vb{B})
    \end{split}
\end{equation}
Si una carga \(q\) está sometida a una fuerza eléctrica \(\vb{E}\) y magnética \(\vb{B}\),
entonces
\begin{equation}
    \begin{split}
        \vb{F} = q(\vb{E} + \vb{v} \times \vb{B})
    \end{split}
\end{equation}
que es la \textbf{fuerza de Lorentz}.
\section{Ley de Ampère}
La circulación de un campo \(\vb{B}\) a lo largo de una línea cerrada \(C\) es igual al
producto de la permeabilidad magnética de medio \(\mu_{0}\) por la suma de corriente que
atraviesa la superfície limitada por el contorno \(C\).
\begin{equation}
    \begin{split}
        \oint_{C} \vb{B} \cdot \dd{\vb{l}} = \mu_{0} \sum  I_{i}
    \end{split}
\end{equation}
\section{Ley de Biot-Savart}
El campo magnético creado por un elemento de corriente que transporta
una corriente estacionaria en un punto $P$ es un vector perpendicular
por el elemento de corriente $\dd{\vb{l}}$ y el vector posición
$\vb{R} = \vb{r} - \vb{r}'$ está dado por
\begin{equation}
	\begin{split}
		\dd{\vb{B}} = \frac{\mu_{0}}{4 \pi} I\frac{(\dd{\vb{l}}\times \vb{R})}{R^{3}}
	\end{split}
\end{equation}
\section{Flujo magnético}
El flujo que atraviesa una superficie es
\begin{equation}
	\begin{split}
		\phi = \oint \vb{B} \cdot \dd{\vb{s}}
	\end{split}
\end{equation}
\subsection{Ley de Gauss para el campo magnético}
El flujo magnético a través de una superfície cerrada es siempre nulo
\begin{equation}
	\begin{split}
		\oint_{S} \vb{B} \cdot \dd{\vb{s}} = 0
	\end{split}
\end{equation}
\section{Relación entre un imán y un circuito eléctrico}
Todo circuito es un dipolo magnético. Similar al momento dipolar, tenemos el momento
dipolar magnético. Este es directamente proporcional a la superfície del contorno,
y se define como:
\begin{equation}
	\begin{split}
		| \vb{m}| = I S
	\end{split}
\end{equation}
El origen del vector es el centro de la espira, y la dirección es la recta perpendicular
a la superfície de la espira. El sentido es saliente si la intensidad va en sentido
antihorario, o entrante si va en sentido horario.\\
Si introducimos una espira paralelo a un campo magnético constante, el sur de la espira va
a ir al norte del campo magnético y viceversa. El momento dipolar magnético siempre se
acaba alineando con el campo magnético. Esto produce el siguiente torque:
\begin{equation}
	\begin{split}
		\vb{\tau} = \vb{m} \times \vb{B}
	\end{split}
\end{equation}
Cuando un electrón se encuentra en un núcleo, el momento es
\[
	|\vb{m}| = \frac{e \pi r ^{2}}{T}
\]
Donde $T$ es el período de la órbita. 
\section{Vector imanación o magnetización}
Se define como
\begin{equation}
	\begin{split}
		\vb{M} = \frac{1}{V} \sum \vb{m}_{i} (\frac{A}{m})
	\end{split}
\end{equation}
\section{Inducción magnética}
\subsection{Ley de Faraday}
En un circuito se induce f.e.m. $\varepsilon$ proporcional a la variación del flujo a
través de la superficie limitada por el propio circuito.
\subsection{Ley de Lenz}
La corriente inducida crea un campo que tiende a conservar el flujo, es decir, el sentido
de la corriente de f.e.m. tiende a oponerse a la variación de flujo:
\begin{equation}
	\begin{split}
		\varepsilon = -\dv{\phi}{t}
	\end{split}
\end{equation}
\begin{equation}
	\begin{split}
		\oint_{C} \vb{E} \cdot \dd{\vb{l}} = -\dv{t}\int_{S} \vb{B}\cdot \dd{\vb{s}}
	\end{split}
\end{equation}
\end{document}
