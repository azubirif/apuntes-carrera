%! TEX root = Probabilidad.tex

\documentclass{./Probabilidad.tex}

\begin{document}
\chapter{Probabilidad condicionada}
- Modelos dinámicos: experimentos que se realizan a lo largo del tiempo. Los experimentos son dependientes entre sí.\\
- Modelos estáticos: mismo período de tiempo.

La probabilidad de que ocurra $B$ si ocurre $A$ es
$$
P(B|A)= \frac{P(A\cap B)}{P(A)}
$$
\section{Regla del producto}
Supongamos un modelo dinámico donde ocurren sucesos de forma consecutiva $\{ A_{1},\dots,A_{n} \}$, se cumple que
\begin{equation}
	\begin{split}
		&P(A_{1}\cap\dots \cap A_{n})=\\ &P(A_{1})P(A_{2}|A_{1})P(A_{3}|A_{1}\cap A_{2})\dots P(A_{n}|A_{1}\cap\dots \cap A_{n-1})
	\end{split}
\end{equation}
\begin{defin}

$A$ y $B$ son sucesos independientes si
$$
P(A\cap B)=P(A)\cdot P(B)
$$

\end{defin}
\begin{defin}
Sea $H$ una familia de sucesos que no incluyen al espacio vacío. Se dice que $H$ está formada por sucesos completamente independientes, si para todo subconjunto de $H$ se verifica que $H=\{ \Delta_{1},\dots,\Delta_{k} \}$
$$
P(\Delta_{1}\cap\dots \cap\Delta_{n})=\prod_{i=1}^j P(\Delta_{i})
$$
\end{defin}
\begin{teorema}
	Dado un modelo matemático probabilístico representado por $(\Omega, \mathcal{A}, P)$, un sistema completo formado por $\{ A_{1},\dots,A_{n} \}\in \mathcal{A}$ (con todos ellos excluyentes), y sabiéndose $\exists B \in \mathcal{A}P(B|A_{i})\forall i$, se verifica que
$$
P(B)=\sum P(A_{i}) P(B|A_{i})
$$
Entonces se verifica que
$$
P(A_{i}|B) = \frac{P(A_{i})P(B|A_{i})}{P(B)}
$$
\end{teorema}
\section{Independencia condicional}
Si se verifica que
$$
P(A\cap B|C)=P(A|C)P(B|C)
$$
Entonces $A$ y $B$ son condicionalmente independientes dado $C$.

\end{document}
