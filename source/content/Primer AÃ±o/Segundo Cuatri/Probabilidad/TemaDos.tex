%! TEX root = Probabilidad.tex

\documentclass{./Probabilidad.tex}

\begin{document}
\chapter{Combinatoria}
Supongamos que tenemos $m$ elementos, y vamos a generar grupos. Cada grupo contiene $n$ elementos.
\section{Variaciones}
\subsection{Variaciones con repetición}
Se llaman variaciones de repetición de $m$ elementos, tomados de $n$ en $n$, a los distintos grupos que se pueden formar de tal manera que cada grupo contiene $n$ elementos distintos o iguales, y que un grupo se diferencia de los demás o bien en algún elemento o su posición.
El número total es
$$
VR(m,n)=m^{n}
$$
\textbf{Ejemplo}: una subrutina genera $1$ byte de forma aleatoria en cada ejecución. La cantidad del espacio muestral es:\\
\begin{itemize}
	\item Como tenemos $m=2$ elementos $(0,1)$, y los tomamos de $8$ en $8$, la cantidad de elementos en el espacio muestral es $V_{8}^{2}=8^{2}=64$.     
\end{itemize}
\subsection{Variaciones sin repetición}
Se les llama, de $m$ elementos, tomados de $n$ en $n$, a los diferentes grupos que se pueden formar tal que
\begin{itemize}
	\item Cada grupo contiene $n$ elementos \textbf{distintos}.
	\item Cada grupo se diferencia de los demás en algún elemento o en su orden.
\end{itemize}
$$
V(m,n)=\frac{m!}{(m-n)!}
$$
\section{Permutaciones}
\subsection{Permutaciones ordinarias}
Si se tienen permutaciones de $n$ elementos, buscamos los distintos grupos que se pueden formar ordenando de diferentes formas $n$ elementos:
$$
P(n)=n!
$$
\subsection{Permutaciones con repetición}
Donde el primer elemento se repite $\alpha$ veces, el segundo $\beta$ veces, y así hasta el último que se repite $\gamma$ veces. La suma $\alpha+\beta+\dots+\gamma=n$. A los distintos grupos de $n$ elementos que se pueden formar con los $m$ elementos, donde cada grupo se compone de $\alpha$ veces el primer elemento, $\beta$ veces el segundo, etc, el número total es
$$
PR(n)= \frac{n!}{\alpha!\beta!\dots\gamma!}
$$
\section{Combinaciones}
\subsection{Combinaciones sin repetición}
De $m$ elementos, tomados de $n$ a $n$, a los diferentes grupos que se pueden formar de tal manera que cada grupo contenta $n$ elementos distintos (el orden no importa). La cantidad total es
$$
C(m,n) = \begin{pmatrix}
m \\ n
\end{pmatrix} = \frac{m!}{n!(m-n)!}
$$
\subsection{Combinaciones con repetición}
De $m$ elementos, los diferentes grupos que se pueden formar tal que cada grupo contiene $n$ elementos distintos, en cada grupo se pueden repetir elementos, y cada grupo se diferencia al menos en un elemento. El número total es:
$$
CR(m,n) = \begin{pmatrix}
m+n-1 \\ n
\end{pmatrix} = \frac{(m+n-1)!}{n!(m-1)!}
$$
\section{Diferencias entre ellas}
Las diferencias entre estas operaciones son las siguientes:
\begin{itemize}
	\item \textbf{Variaciones}: el orden SÍ importa.
	\item \textbf{Combinaciones}: el orden NO importa.
	\item \textbf{Permutaciones}: caso especial de variaciones donde ordenamos todos los elementos del conjunto $(n=m)$. 
\end{itemize}
\end{document}
