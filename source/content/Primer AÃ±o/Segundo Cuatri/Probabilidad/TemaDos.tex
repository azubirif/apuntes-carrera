%! TEX root = Probabilidad.tex

\documentclass{./Probabilidad.tex}

\begin{document}
\chapter{Combinatoria}
Supongamos que tenemos $m$ elementos, y vamos a generar grupos. Cada grupo contiene $n$ elementos.
\section{Variaciones}
\subsection{Variaciones con repetición}
Se llaman variaciones de repetición de $m$ elementos, tomados de $n$ en $n$, a los distintos grupos que se pueden formar de tal manera que cada grupo contiene $n$ elementos distintos o iguales, y que un grupo se diferencia de los demás o bien en algún elemento o su posición.
El número total es

$$
VR_{m}^n=m^{n}
$$
\subsection{Variaciones sin repetición}
Se les llama a las diferentes formas en las que se pueden ordenar $m$ elementos formados en grupos de $n$ elementos. El número total es
$$
V_{m,n}=\frac{m!}{(m-n)!}
$$
\subsection{Permutaciones ordinarias}
Si se tienen permutaciones de $n$ elementos, buscamos los distintos grupos que se pueden formar ordenando de diferentes formas $n$ elementos:
$$
P_{n}=n!
$$
\subsection{Permutaciones con repetición}
Donde el primer elemento se repite $\alpha$ veces, el segundo $\beta$ veces, y así hasta el último que se repite $\gamma$ veces. La suma $\alpha+\beta+\dots+\gamma=n$. A los distintos grupos de $n$ elementos que se pueden formar con los $m$ elementos, donde cada grupo se compone de $\alpha$ veces el primer elemento, $\beta$ veces el segundo, etc, el número total es
$$
P_{n}= \frac{n!}{\alpha!\beta!\dots\gamma!}
$$
\subsection{Combinaciones ordinarias o sin repetición}
De $m$ elementos, tomados de $n$ a $n$, a los diferentes grupos que se pueden formar de tal manera que cada grupo contenta $n$ elementos distintos (el orden no importa). La cantidad total es
$$
C_{m,n} = \begin{pmatrix}
m \\ n
\end{pmatrix} = \frac{m!}{n!(m-n)!}
$$
\subsection{Combinaciones con repetición}
De $m$ elementos, los diferentes grupos que se pueden formar tal que cada grupo contiene $n$ elementos distintos, en cada grupo se pueden repetir elementos, y cada grupo se diferencia al menos en un elemento. El número total es:
$$
CR_{m,n} = \begin{pmatrix}
m+n-1 \\ n
\end{pmatrix} = \frac{(m+n-1)!}{n!(m-1)!}
$$
\end{document}
