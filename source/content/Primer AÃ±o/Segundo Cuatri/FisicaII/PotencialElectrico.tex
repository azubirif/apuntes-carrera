%! TEX root = FisicaII.tex

\documentclass{./FisicaII.tex}

\begin{document}
\chapter{Potencial Eléctrico}
\section{Trabajo y energía potencial}
Supongamos una carga $q_{2}$, sometida a un determinado campo eléctrico $\mathbf{E}$, generado por una carga $q_{1}$ (en reposo). La carga $q_{2}$ se sitúa inicialmente  en $\mathbf{r}_{i}$, y se desplaza, bajo el efecto del campo eléctrico, hasta $\mathbf{r}_{f}$.\\
Este trabajo, aplicando la ley de Coulomb, termina siendo
$$
W=\int_{r_{i}}^{r_{f}} \frac{1}{4\pi\varepsilon_{0}} \frac{q_{1}q_{2}}{r^2}dr = -\left( \frac{q_{1}q_{2}}{4\pi\varepsilon_{0}r_{f}} - \frac{q_{1}q_{2}}{4\pi\varepsilon_{0}r_{i}} \right)=U_{i}-U_{f}=-\Delta U
$$
de aquí se deduce que el trabajo depende de la función **energía potencial** $U$. Esta depende únicamente de la posición de la partícula $q_{2}$ en cada momento $t$, y viene dada por

$$
\boxed{
U=\frac{1}{4\pi\varepsilon_{0}} \frac{q_{1}q_{2}}{|r|}
}
$$
El trabajo realizado al desplazar la carga de $\mathbf{r}_{i}$ a $\mathbf{r}_{f}$ se puede expresar como la diferencia de la energía potencial:
$$
W=-\Delta U
$$
De aquí se puede deducir que el trabajo realizado para desplazar no depende de la trayectoria, sino únicamente de lo punto inicial y final.
\section{Potencial eléctrico}
Si la carga receptora $q_{2}$ está sometida a un campo eléctrico $\mathbf{E}$, se cumple que la fuerza generada es $\mathbf{F}=q_{2}\mathbf{E}$. Por tanto, el trabajo realizado por el campo eléctrico al desplazar $q_{2}$ se puede definir como
\begin{equation}
	\begin{split}
W&=\int_{r_{i}}^{r_{f}}q_{2}\mathbf{E}\cdot \mathbf{dr}\\
\frac{W}{q_{2}} &= \int_{r_{i}}^{r_{f}}\mathbf{E}\cdot \mathbf{dr}\\
\frac{W}{q_{2}}&= \frac{q_{1}}{4\pi\varepsilon_{0}r_{i}}-\frac{q_{1}}{4\pi\varepsilon_{0}r_{f}}=-\Delta V
	\end{split}
\end{equation}
por tanto, definimos el potencial eléctrico como
$$
\boxed{
V(r) = \frac{q_{1}}{4\pi\varepsilon_{0}r}~ \left( \frac{J}{C} \right) = \frac{U}{q_{2}}~(V)
}
$$
que es la energía potencial por unidad de carga. También podemos deducir que
$$
[E]=\left[ \frac{V}{m} \right]
$$
Si sabemos la expresión de un campo elécitro, se da que
\begin{equation}
	\begin{split}
		V(\vb{r}) - V(\infty) = -\int_{\infty}^{\vb{r}}\vb{E}\cdot \dd{\vb{r}}
	\end{split}
\end{equation}
\section{Campo vectorial}
Supongamos una carga $q_{1}$ que genera un campo eléctrico $\mathbf{E}$ sobre una carga de $1~(C)$ a una distancia $r$ del foco. Se dice que $\mathbf{E}$ es conservativo si cumple que:\\
- $\mathbf{E}$ solo depende de la posición, y no depende explícitamente del tiempo.\\
- Si se cumple $1$, existe una función $V(\mathbf{r})$ tal que $V(\mathbf{r}_{i})\\
- V(\mathbf{r}_{f})=\frac{W}{q_{2}}$. La diferencia de potencial solo depende de la posición inicial y final de la carga desplazada.\\
- La integral del trabajo a lo largo de un camino cerrado es $0$. Consecuentemente, la circulación de un vector siempre es nula.\\
- $dV=-\mathbf{E}\cdot \mathbf{dr}$
\section{Potencial por una distribución de carga}
Supongamos un conjunto de cargas que se denotan $\{ q_{i} \}^{n}$, la posición relativa de cada carga al punto $P$ se denota $\mathbf{r}_{i}$.\\
El potencial en $P$ es la suma de potenciales inviduales:
\begin{equation}
	\begin{split}
		V(\mathbf{r}_{P}) = \sum_{i=1}^{n} \frac{1}{4\pi \varepsilon_{0}}
		\frac{q_{i}}{|\mathbf{r}_{i}|}
	\end{split}
\end{equation}
\section{Potencial por distribución de carga continua}
\subsection{Lineal}
El potencial debido a una distribución de carga lineal con densidad de carga
\[
	\lambda = \dv{q}{l}
\]
Al tener una carga infinitesimal $\dd{q}$, le corresponde un potencial infinitesimal $\dd{V}$, por lo que, integrando, tenemos
\begin{equation}
	\begin{split}
		V(r) = \int \frac{1}{4\pi \varepsilon_{0}} \frac{\lambda \dd{l}}{|r|}
	\end{split}
\end{equation}
\subsection{Superficie equipotencial}
Una superfície equipotencial (Figura \ref{fig:superficie-equipotencial}) el conjunto de puntos donde el potencial en cada uno de ellos es el mismo. En una superfície equipotencial, se cumple que $V(\vb{r})=C$, por lo que
\[
	\pdv{V}{r}=0
\]
Las líneas de campo siempre son perpendiculares a la superfície equipotencial.
\begin{figure}[ht]
    \centering
    \incfig{superficie-equipotencial}
    \caption{Superfície Equipotencial}
    \label{fig:superficie-equipotencial}
\end{figure}
\end{document}
