\documentclass{report}
\author{Alejandro Zubiri}
\title{Física II}

\renewcommand*\contentsname{Índice}

\usepackage[margin=1.1in]{geometry}
\usepackage{amsmath, physics, amsthm, amsfonts, mdframed, subfiles, tikz, hyperref, fancyhdr, silence}

\pagestyle{fancy}
\fancyfoot[R]{Página \thepage}
\fancyfoot[L]{Alejandro Zubiri}

\WarningsOff[hyperref]
\WarningsOff[fancyhdr]

\newmdtheoremenv{teorema}{Teorema}
\newmdtheoremenv{defin}{Definición}

\newcommand{\R}{\mathbb{R}}

\usepackage{import}
\usepackage{pdfpages}
\usepackage{transparent}
\usepackage{xcolor}

\newcommand{\incfig}[2][1]{%
    \def\svgwidth{#1\columnwidth}
    \import{./figures/}{#2.pdf_tex}
}

\pdfsuppresswarningpagegroup=1

\begin{document}
\maketitle
\begin{equation}
	\begin{split}
		\nabla \cdot \vec{E} &= \frac{q}{\varepsilon_{0}}\\
		\nabla \cdot \vec{B} &= 0
	\end{split}
\end{equation}
\pagebreak
\tableofcontents
\pagebreak
\section{Bibliografía}
\begin{itemize}
	\item Física para la ciencia y la tecnología - Tipler y Mosca
	\item Física, Volumen II - Campos y Ondas - M. Alonso. E. J. Finn
\end{itemize}
\subfile{CampoElectrico.tex}
\subfile{PotencialElectrico.tex}
\end{document}
