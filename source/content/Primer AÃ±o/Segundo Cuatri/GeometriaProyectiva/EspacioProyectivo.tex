%! TEX root = Geometria.tex

\documentclass{Geometria.tex}

\begin{document}
\chapter{Espacio proyectivo}
\section{Introducción}
Todas las rectas paralelas se cortan en el infinito. Supongamos que tenemos dos planos
\(\Pi_{1}\) y \(\Pi_{2}\), y una aplicación \(\phi \) que lleva puntos del primer plano al segundo
como la intersección de la recta que va de un punto del plano hasta un origen \(O\) y el segundo
plano.\\
Definimos el plano proyectivo de \(\mathbb{R}^{3}\) como
\[
    \mathbb{P}(\mathbb{R}^{3}) = \mathbb{R}^{2} - \{ 0 \} + P_{\infty}
\]
Tenemos que
\begin{itemize}
    \item Dos rectas siempre se cortan, aunque sea en el infinito.
\end{itemize}
Un punto en el plano proyectivo se define por sus coordenadas homogéneas, como un vector de una
recta. Se refleja en el plano afín con las coordenadas no homogéneas \((1, \frac{x_{1}}{x_{0}}
, \frac{x_{2}}{x_{0}})\) si \(x_{0} \neq 0\). Sino, como el vector \(0, x_{1}, x_{2}\).
\section{Espacio afín y proyectivo}
Si tenemos un espacio afín \(A\) de dimensión \(n\), entonces los puntos \(X\) se pueden ver como
puntos de \(\mathbb{P(R^{n+1})}\) como
\[
    A \to  \mathbb{P}(\mathbb{R}^{n+1})
\]
\[
    X \to (1: X)
\]
Los puntos que no son de la forma \((1:X)\) son puntos en el infinito o impropios.
\begin{defin}
    Se define una relación de equivalencia entre dos vectores si estos pertenecen a la misma recta.
    Se define el plano proyectivo como el conjunto cociente \(\frac{R^{3} - \{ 0 \}}{\sim}\).
\end{defin}
\begin{defin}
    Dado un espacio proyectivo \(\mathbb{P}(V)\), se llama subespacio proyectivo de dimensión \(k\)
    a un subconjunto de la forma \(\mathbb{P}(W)\) donde \(W\) es un subespacio vectorial
    de \(V\), de dimensión \(k+1\).
\end{defin}
\begin{defin}
    Un sistema de referencia proyectivo se define por \(k+2\) puntos y una base normalizada. Tenemos
    el punto unidad definido como
    \[
        P_{k+1} = \sum _{1}^{n} P_{i}
    \]
\end{defin}
\begin{defin}
    El espacio afín completado se define como
    \[
        \mathbb{P}^n(\mathbb{R}^{n+1}) = A_{n} \cup S_{\infty}
    \]
\end{defin}
Una recta en el espacio proyectivo es una recta afín más un punto en el infinito.
\begin{defin}
    Una proyectividad de \(\mathbb{P}^{2} (\mathbb{R})\) es una aplicación \(f: \mathbb{P}^{2}(\mathbb{R})
    \to  \mathbb{P}^{2}(\mathbb{R})\) dada por
    \[
        f([P]) = [\bar{f}(P)]
    \]
    \(\forall P \in \mathbb{R}^{3}\), con \(\bar{f}: \mathbb{R}^{3} \to  \mathbb{R}^{3}\) un
    isomorfismo vectorial, \(\bar{f}(P) = AP\).
\end{defin}
\section{Ecuaciones de las rectas del plano proyectivo}
Dados dos puntos independientes $P,Q \in \mathbb{P} ^{2}$, se tiene que
$P = <\vv{v}>$ y que $Q = <\vv{w}>$, siendo estos vectores de $\mathbb{R}^{3}$
linealmente independientes. La recta $r$ que contiene a $P,Q$ es
\[
	r = \{ <\lambda \vv{v} + \mu \vv{w}>: (\lambda,\mu) \neq (0,0) \}
\]
\subsection{Ecuaciones paramétricas}
Sean los puntos $P,Q$ con coordenadas homogéneas. Se tiene que un punto $X$
pertenece a la recta $r$ si y solo si sus coordenadas cumplen las ecuaciones
\[
	\alpha x_{i} = \lambda p_{i} + \mu q_{i} : (\alpha, \lambda, \mu)
	\neq (0,0,0)
\]
\subsection{Ecuaciones implícitas}
Simplemente calculamos
\[
	\det (X, P, Q) = 0
\]
Teniendo una recta propia
\[
	a_{0}x_{0} + a_{1}x_{1} + a_{2}x_{2} = 0
\]
El punto impropio tiene coordenadas $(-a_{2}, a_{1})$. La intersección de dos rectas es el producto vectorial de sus componentes. 
\section{Planos}
Sean tres puntos $P,Q,R$ de $\mathbb{P} ^{3}$ linealmente independientes. El plano que los contiene es el formado por el subespacio vectorial de los vectores asociados a cada punto.  
\section{Notas para el examen}
\begin{itemize}
  \item Las rectas afines se mappean a puntos proyectivos,
    los planos afines a rectas proyectivas etc.
  \item El punto impropio de una recta es $[0: V]$ donde $V$
    es el vector director de la recta.
  \item Teniendo un plano, podemos sacar su recta proyectiva
    multiplicando su término independiente por una nueva variable.
\end{itemize}
\end{document}
