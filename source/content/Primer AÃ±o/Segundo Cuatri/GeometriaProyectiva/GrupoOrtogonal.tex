%! TEX root = Geometria.tex

\documentclass{./Geometria.tex}

\begin{document}
\chapter{Grupo ortogonal}
\textbf{Resumen}:
\begin{itemize}
    \item Ortogonalidad; \(\vb{x} \cdot \vb{y} = 0\)
    \item Base ortogonal: \(\vb{v}_i \cdot \vb{v}_j = 0\)
    \item Base ortonormal: \(\vb{v}_i \cdot \vb{v}_j = 0, \norm{\vb{v}} = 1\)
    \item Matriz regular: cuadrada, invertible, \(\det \neq 0\)
    \item Matriz simétrica: \(A^{T} = A\)
    \item Matriz ortogonal: \(A^{T} = A^{-1}\)
    \item \(A,B\) congruentes, \(A = P^{T}BP\)
    \item \(A,B\) semejantes, \(A = P^{-1}BP\)
    \item Sis. referencia rectangular, con base ortonormal.
    \item Para todo espacio euclídeo, existe una base ortogonal (Método de Gram-Schmidt) y una base ortonormal.
\end{itemize}
\begin{defin}
    Decimos que una matriz es ortogonal si y solo si
    \[
        P^{T}P = I
    \]
    El determinante de una matriz de este tipo es siempre \(\pm 1\).
\end{defin}
\begin{teorema}
    Las matrices de aplicaciones ortogonales respecto a bases ortonormales son matrices
    ortogonales.
\end{teorema}
\begin{defin}
    Sea \(f\) un endomorfismo de un EVE \(V\). Se dice que \(f\) es simétrico si y solo si
    \[
        \vb{x} \cdot f(\vb{y}) = f(\vb{x})\cdot \vb{y}~ \forall \vb{x}, \vb{y} \in V
    \]
\end{defin}
\begin{defin}
    La matriz de la proyección ortogonal de un subespacio vectorial \(W\) es
    \[
        P = A(A^{T}A)^{-1}A^{T}
    \]
    Siendo \(A\) la matriz cuyas columnas son las coordenadas de una base de \(W\).
\end{defin}
\begin{defin}[Matriz de Householder]
    Sea el vector \(\vb{u} \in \mathbb{R}^{n}\) no nulo. Se define la matriz
    de House holder asociada a \(\vb{u}\) \(H(\vb{u})\) como
    \[
        H(\vb{u}) = I - 2 \frac{\vb{u}^{T} \vb{u}}{\vb{u} \vb{u}^{T}}
    \]
    Esta matriz transforma un vector de forma que lo refleja sobre un subespacio.
\end{defin}
\section{Factorización QR}
Dada una matriz \(A\), con columnas linealmente independientes, encontramos matrices
\(Q,R\) tales que
\begin{itemize}
    \item \(A=QR\)
    \item Las columnas de \(Q\) son ortonormales.
    \item \(Q\) es del mismo tamaño que \(A\).
    \item \(R\) es triangular superior invertible.
\end{itemize}
\end{document}