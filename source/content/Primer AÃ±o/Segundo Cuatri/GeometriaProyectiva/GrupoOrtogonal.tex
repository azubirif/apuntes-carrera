%! TEX root = Geometria.tex

\documentclass{./Geometria.tex}

\begin{document}
\chapter{Grupo ortogonal}
\textbf{Resumen}:
\begin{itemize}
    \item Ortogonalidad; \(\vec{x} \cdot \vec{y} = 0\)
    \item Base ortogonal: \(\vec{v}_i \cdot \vec{v}_j = 0\)
    \item Base ortonormal: \(\vec{v}_i \cdot \vec{v}_j = 0, \norm{\vec{v}} = 1\)
    \item Matriz regular: cuadrada, invertible, \(\det \neq 0\)
    \item Matriz simétrica: \(A^{T} = A\)
    \item Matriz ortogonal: \(A^{T} = A^{-1}\)
    \item \(A,B\) congruentes, \(A = P^{T}BP\)
    \item \(A,B\) semejantes, \(A = P^{-1}BP\)
    \item Sis. referencia rectangular, con base ortonormal.
    \item Para todo espacio euclídeo, existe una base ortogonal (Método de Gram-Schmidt) y una base ortonormal.
\end{itemize}
\end{document}