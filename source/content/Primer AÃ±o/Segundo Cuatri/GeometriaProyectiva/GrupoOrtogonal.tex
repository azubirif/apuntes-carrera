%! TEX root = Geometria.tex

\documentclass{./Geometria.tex}

\begin{document}
\chapter{Grupo ortogonal}
\textbf{Resumen}:
\begin{itemize}
    \item Ortogonalidad; \(\vec{x} \cdot \vec{y} = 0\)
    \item Base ortogonal: \(\vec{v}_i \cdot \vec{v}_j = 0\)
    \item Base ortonormal: \(\vec{v}_i \cdot \vec{v}_j = 0, \norm{\vec{v}} = 1\)
    \item Matriz regular: cuadrada, invertible, \(\det \neq 0\)
    \item Matriz simétrica: \(A^{T} = A\)
    \item Matriz ortogonal: \(A^{T} = A^{-1}\)
    \item \(A,B\) congruentes, \(A = P^{T}BP\)
    \item \(A,B\) semejantes, \(A = P^{-1}BP\)
    \item Sis. referencia rectangular, con base ortonormal.
    \item Para todo espacio euclídeo, existe una base ortogonal (Método de Gram-Schmidt) y una base ortonormal.
\end{itemize}
\begin{defin}
    Decimos que una matriz es ortogonal si y solo si
    \[
        P^{T}P = I
    \]
    El determinante de una matriz de este tipo es siempre \(\pm 1\).
\end{defin}
\begin{teorema}
    Las matrices de aplicaciones ortogonales respecto a bases ortonormales son matrices
    ortogonales.
\end{teorema}
\begin{defin}
    Sea \(f\) un endomorfismo de un EVE \(V\). Se dice que \(f\) es simétrico si y solo si
    \[
        \vec{x} \cdot f(\vec{y}) = f(\vec{x})\cdot \vec{y}~ \forall \vec{x}, \vec{y} \in V
    \]
\end{defin}
\begin{defin}
    La matriz de la proyección ortogonal de un subespacio vectorial \(W\) es
    \[
        P = A(A^{T}A)^{-1}A^{T}
    \]
    Siendo \(A\) la matriz cuyas columnas son las coordenadas de una base de \(W\).
\end{defin}
\end{document}