%! TEX root = Calculo.tex

\documentclass{./Calculo.tex}

\begin{document}
\chapter{La integral de Lebesgue}
\section{Sobre la integral de Riemann}
Para integrar una función $f$ en un intervalo $[a,b]$, introducíamos el concepto de partición de $[a,b]$:
\[
	P = \{ x_0=a,x_1,\dots ,x_{n} = b \}
\]
Una partición de $[a,b]$ es un conjunto finito de puntos ordenados. También necesitamos el ínfimo y el supremo de una funcińo en un conjunto:
\[
	\inf f = \inf \{ f(x) : x \in P \}
\]
\[
	\sup f = \sup \{ f(x) : x \in P \}
\]
Con esto vamos a construir las \textbf{sumas de Riemann}, superiores e inferiores, denotadas por $S$ y $s$, respectivamente, y se definen como:
dada $f$ definida en $[a,b]$  introducimos una partición $P$ de este intervalo, y construimos
\[
	s = \sum_{i}^{n} (x_{i} - x_{i-1}) \inf f
\]
es la suma inferior, y
\[
	S = \sum_{i}^{n} (x_{i}-x_{i-1}) \sup f
\]
Esto define el área de la curva bajo la función $f$ en el intervalo $[a,b]$. Si tomamos el límite $n \to \infty$, vemos que $S = s$, y entonces
\[
	\int_{b}^{a} f = \lim_{n \to \infty} \sum (x_{i}-x_{i-1}) \sup f
\]
\begin{teorema}
	Una función $f(x)$ es integrable de Riemann en $[a,b]$ si es continua en ese intervalo.  
\end{teorema}
Sin embargo, la integral de Riemann no es suficientemente buena, ya que:
\begin{itemize}
	\item No funciona si hay muchas discontinuidades.
\end{itemize}
\section{Una nueva integral}
Los problemas de la integral de Riemann se deben a querer calcular áreas sobre intervalos, pero las funciones integrables Riemann han de ser lo suficientemente buenas en dichos intervalos.\\
En vez de considerar solo la medida de intervalos, vamos a fijarnos en los valores que toma $f$, y entonces medir las preimágenes:
\[
	f^{1-}(\{ a \}) = \{ x : f(x) = a \}
\]
De manera informal, queremos
\[
	f: [a,b] \subset \mathbb{R} \to \mathbb{R} : f \geq 0 \forall x \in [a,b]
\]
para ello, introducimos una partición en la \textbf{imagen} de $f$. 
\[
	P : 0 = y_0 < y_1 <\cdots < y_{n} = \sup f
\]
Construimos entonces la suma inferior de Lebesgue:
\[
	\mathcal{L}(f, P, [a,b]) = \sum _{j=1}^{n} \mu (f^{-1}([y_{j-1}, y_{j}])) \cdot y_{j-1}
\]
Donde $\mu$ es la \textbf{medida} del intervalo. La medida que usaremos será una generalización de la medida de longitud de intervalos abiertos de la forma $(a,b) \subset \mathbb{R}$:
\[
	l((a,b)) = b-a
\]
además de que $l(\phi) = 0$. También 
\[
	l(-\infty, a) = l(a,\infty) = = l(-\infty, \infty) = \infty
\]
La generalización para medir cualquier $A \subset \mathbb{R}$ es:
\[
	|A| = \inf \{ \sum_{k=1}^{\infty}l(I_{k}) : I_{k} \text{ son intervalos abiertos tales que }A \subset \bigcup I_{k} \}
\]
Esta medida no será la que finalmente usemos, ya que existen $A, B \subset  \mathbb{R}$, disjuntos, y para los cuales:
\[
	|A \cup B| \neq |A| + |B|
\]
De hecho no existe ninguna medida que cumpla las siguientes propiedades que refleje la idea intuitiva de tamaño:
\begin{itemize}
	\item $\mu: \mathcal{P}(\mathbb{R}) \to [0, \infty]$ 
	\item Si $\{ A_{k} \}$ son disjuntos entonces 
		\[
			\mu(\bigcup A_{k} ) = \sum |A_{k}|
		\]
	\item Si $A \subset \mathbb{R}, t \in \mathbb{R}$ y 
		\[
			t+ A = \{ t+a : a \in A \}
		\]
		entonces esperamos que $\mu(A) = \mu(A+t)$ 
	\item Si $I$ es un intervalo abierto entonces
		\[
			\mu(I) = l(I)
		\]
\end{itemize}
Por ello, vamos a eliminar la primera propiedad, usando la medida exterior, y nos restringiremos a una familia de conjuntos conocidos como los \textbf{conjuntos de Borel}.
\begin{defin}
	Un conjunto de Borel es de la familia $\mathcal{B}$ mínima (en sentido de inclusión) que contiene a todos los intervalos abiertos de $\mathbb{R}$ y que cumple:
	\begin{enumerate}
		\item Cerrada bajo uniones numerables.
		\item Cerrada bajo complementos.
	\end{enumerate}
\end{defin}
Con esto, observamos que $\phi$ es un boreliano.
\section{Medida de Lebesgue}
\begin{defin}
	Cuando restringimos la medida exterior al conjunto de los borelianos, la medida pasa a llamarse \textbf{medida de Lebesgue}, y la denotaremos por $\mu$. 
	\[
		\mu: \mathcal{B} \to [0,\infty]
	\]
	Esta medida cumple:
	\begin{enumerate}
		\item $\mu(I) = l(I)$ si $I$ es un intervalo.  
		\item $\mu(t+B) = \mu(B)$ para todo $t \in \mathbb{R}$ y todo $B \in \mathcal{B}$.  
		\item $\mu(\bigcup B_{k}) = \sum \mu (B_{k})$ con $\{ B_{k} \} \in \mathcal{B}$ y disjuntos. 
	\end{enumerate}
\end{defin}
Todos los conjuntos numerables tienen medida $0$. \\
\begin{defin}
Una propiedad $P(x)$ con $x \in \mathbb{R}$, decimos que es casi doquier, abreviado como c.d., si  
\[
	\mu(\{ x \in \mathbb{R} : P(x) \text{ es falsa} \}) = 0
\]
\end{defin}
\begin{defin}[Conjuntos medibles Lebesgue]
	Son aquellos conjuntos a los que se puede extender $\mu$. Si $L$ es medible Lebesgue entonces existe un boreliano $B$ tal que   
	\[
		B \subset L \wedge \mu(L-B) = 0
	\]
\end{defin}
\begin{defin}[Funciones medibles Borel]
	$f : D \subset \mathbb{R} \to \mathbb{R}$ es medible Borel si $\forall B \in \mathcal{B}$
	\[
		f^{-1}(B) \in \mathcal{B}
	\]
	En particular, como $R \in \mathcal{B}$, entonces $f^{-1}(\mathbb{R}) = D \in \mathcal{B}$.\\
	Es interesante hacer notar el paralelismo entre conjuntos abiertos y funciones continuas, y conjuntos borelianos y funciones medibles de Borel.
\end{defin}
A las funciones medibles Borel también se las llama borelianas. Esta definición se extiende también a funciones que toman valores infinitos, donde exigimos que
\[
	f^{-1}(\{ -\infty \}) \in \mathcal{B} \wedge f^{-1}(\{ \infty \}) \in \mathcal{B}
\]
Para identificar si una función es boreliana, en vez de usar la definición, podemos usar alguno de los siguientes criterios:
\begin{teorema}
	$f: D \subset  \mathbb{R} \to \mathbb{R}$ entonces 
	\begin{itemize}
		\item $f$ es boreliana si y solo si 
			\[
				f^{-1}((a,b)) \in \mathcal{B}
			\]
		\item $f$ es boreliana si y solo si 
			\[
				f^{-1}((a, \infty)) \in \mathcal{B}
			\]
		\item $f$ es boreliana si y solo si 
			\[
				\{ x \in D : f(x) < b  \} \in \mathcal{B}
			\]
		\item Si $\{ f_{n} \}$ son borelianos tales que 
			\[
				\lim_{n \to \infty} f_{n} = f(x)~c.d.
			\]
			entonces $f$ es boreliana. 
		\item Si $f$ es continua en $D$ y $D$ es boreliano entonces $f$ es boreliana.    
	\end{itemize}
\end{teorema}
\section{Integración de Lebesgue}
Para integrar según Lebesgue una función boreliana no negativa $f: D \subset \mathbb{R} \to [0,\infty]$, dividimos la imagen de $f$ mediante una partición $P = \{ y_0,\dots ,y_{n} \}$ (ordenada), formando las sumas inferiores de Lebesgue:
\[
	\mathcal{L}(f,P) = \sum y_{k-1}\mu(f^{-1}([y_{k+1},y_{k}]))
\]
La integral de Lebesgue de $f$ en $D$ es entonces:
\begin{equation}
	\begin{split}
		\int_{D}f \dd{\mu} = \sup \mathcal{L}(f,P)
	\end{split}
\end{equation}
\section{Función integrable Lebesgue}
Dada $f: D \subset \mathbb{R} \to \mathbb{R}$, definimos:
\begin{align}
	f_{+}(x) = \max \{ f(x), 0: x \in D \}
\end{align}
y
\begin{align}
	f_{-}(x) = \max \{ -f(x), 0: x \in D \}
\end{align}
De esta forma nos aseguramos de estar calculando el área bajo la función. Decimos
que \(f\) es integrable Lebesgue en \(D\), denotado por
\[
	f \in \mathcal{L}_{1}(D)\footnote{El 1 hace referencia a que integramos el módulo
	elevado a 1.}
\]
si
\[
	\int_{D}|f| \dd{\mu} = \int_{D}f_{+} \dd{\mu} + \int_{D} f_{-} \dd{\mu} < \infty
\]
y la integral de \(f\) se define por
\begin{equation}
	\begin{split}
		\int_{D} f \dd{\mu} = \int_{D} f_{+} \dd{\mu} - \int_{D} f_{-} \dd{\mu}
	\end{split}
\end{equation}
Análogamente se definen las funciones \(f \in  \mathcal{L}_{P}(D)\)
\[
	\int_{D} |f|^{P} \dd{\mu} < \infty : p \in [1,\infty)
\]
Algunas propiedades de esta integral son:
\begin{itemize}
	\item Una función integrable de Riemann es también integrable de Lebesgue.
	\item \(f=g~e.c.t~D \subset \mathbb{R} \implies \int_{D} f \dd{\mu} = \int_{D} g \dd{\mu}\) 
\end{itemize}
Comprobemos, para finalizar, que la integral de Lebesgue no tiene los problemas que
tenía la integral de Riemann.
\begin{itemize}
	\item La integral de Lebesgue puede integrar funciones con infinitas
	discontinuidades (como la función de Dirichlet).
\end{itemize}
\end{document}
