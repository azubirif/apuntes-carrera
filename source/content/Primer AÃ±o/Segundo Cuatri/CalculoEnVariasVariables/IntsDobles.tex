%! TEX root = Calculo.tex

\documentclass{./Calculo.tex}

\begin{document}
\chapter{Integrales múltiples}
\section{Integrales dobles}
\begin{defin}
    Los borelianos en \(\mathbb{R}^{2}\) son la familia mínima que contiene
    a los conjuntos
    \[
        A \times B
    \]
    con \(A,B \in \mathcal{B}(\mathbb{R})\) y que cumple:
    \begin{itemize}
        \item ser cerrada bajo complementación.
        \item ser cerrada bajo uniones numerables.
    \end{itemize}
    A esta familia la denotaremos por \(\mathcal{B}(\mathbb{R}^{2})\) o
    \(\mathcal{B}(\mathbb{R}) \otimes \mathcal{B}(\mathbb{R})\).
\end{defin}
También necesitaremos una medida para estos borelianos, por lo que usaremos la
\textbf{medida producto}. Dado
\[
    A \times B \in  \mathcal{B}(\mathbb{R}^{2})
\]
entonces
\[
    \mu_{2}(A \times B) = \mu(A)\cdot \mu(B)
\]
Observemos que si, por ejemplo, \(A =\{ a \} \subset \mathbb{R}\) entonces
\[
    \mu_{2}(\{ a \} \times B) = 0
\]
Los conjuntos unidimensionales tienen medida \(\mu_{2}\) nula.\\
Primero veamos cómo integrar funciones no negativas. Supongamos que tenemos
una región \(D\) donde queremos integrar una función \(f\). Esta integral nos va
a dar el volumen encerrado entre la superfície \(f\) y el plano \(XY\).
\[
    \iint_{D} f(x,y) \dd{\mu_{2}(x,y)}
\]
Pero esta integral no se suele poder calcular de forma directa. Para ello,
en vez de integrar respecto a todas las variables a la vez, las integraremos una
por una, lo que se conoce como una \textbf{integral iterada}:
\[
    \int_{b}^a \int_{c}^d f(x,y) \dd{\mu(y)} \dd{\mu(x)}
\]
Sin embargo, esto solo tendrá sentido si podemos conmutar y asociar los
diferenciales como queramos. Para saber esto, utilizaremos el siguiente teorema:
\begin{teorema}[Teorema de Fubini]
    Sea \(f: D \subset \mathbb{R}^{2} \to \mathbb{R}\). Si \(f\) es continua e.c.t.
    \(D\) entonces da igual el orden de las integrales iteradas.
    \[
        \iint_{D} f(x,y) \dd{x} \dd{y} = \iint_{D} f(x,y) \dd{y} \dd{x}
    \]
\end{teorema}
\section{Área de un triángulo definido por vectores}
Si tenemos dos vectores \(\vb{v}\) y \(\vb{w}\) en \(\mathbb{R}^{2}\), tenemos que
\[
    A(\vb{v}, \vb{w}) = \frac{1}{2} v_{1} w_{2}
\]
Sin embargo, esta expresión no es covariante (cambia respecto al sistema de coordenadas)
respecto a rotaciones. Por otro lado, sabemos que el área es invariante bajo
transformaciones, busquemos cantidades que sean invariantes bajo rotaciones y que dependan
de \(\vb{v}\) y \(\vb{w}\).
\begin{itemize}
    \item El producto escalar, pero \(\vb{v} \cdot \vb{w} = v_{1}w_{1}+v_{2}w_{2}\), por
    lo que no nos sirve.
    \item La traza de una matriz, por lo que podemos construir
    \[
        M = \mqty(v_{1} & w_{1} \\ 0 & w_{2})
    \]
    pero \(\tr(M) = v_{1} + w_{2}\)
    \item El determinante de \(M\) en este caso es \(\det (M) = v_{1}w_{2}\)
\end{itemize}
Por tanto
\[
    A(\vb{v}, \vb{w}) = | \frac{1}{2}\det (M)| = \frac{1}{2}|v_{1}w_{2}-v_{2}w_{1}|
\]
\section{Cambios de variable}
Dada
\[
    \iint_{D} f(x,y) \dd{x} \dd{y}
\]
nos puede interesar realizar un cambio de variables, es decir, hacer las sustituciones
\[
    x = g_{1}(u,v)
\]
\[
    y = g_{2}(u,v)
\]
denotando \(\vb{g} = (g_{1},g_{2})\), \(\vb{x}=(x,y)\) y \(\vb{u}=(u,v)\), obteniendo
\[
    \vb{x} = \vb{g}(\vb{u})
\]
y sustituyendo
\[
    f(\vb{x}) = f(\vb{g}(\vb{u})) = (f\circ \vb{g})(\vb{u})
\]
Sin embargo, puede que las áreas finales no sean las mismas que las iniciales. Para ello,
necesitamos ver la relación entre \( \dd{x} \dd{y}\) y \( \dd{u} \dd{v}\).\\
Vamos a ver la relación entre el conjunto final y el inicial. Tenmos que
\[
    A(R_{f}) = A(R) + o(\Delta u \Delta v) = \det  \mqty(a_{1} & b_{1} \\ a_{2} & b_{2})
\]
Ahora falta calcular esos componentes:
\[
    \vb{a} = \vb{g}(u+\Delta u, v) - \vb{g}(u,v) = \pdv{\vb{g}}{u}\eval_{(u,v)}\Delta u
\]
y
\[
    \vb{b} = \vb{g}(u, v + \Delta v) - \vb{g}(u,v) = \pdv{\vb{g}}{v}\eval_{(u,v)}\Delta v
\]
Por tanto, el área es
\[
    A(R_{f}) = \det  \mqty(\pdv{g_{1}}{u}\Delta u & \pdv{g_{1}}{v}\Delta v \\ 
    \pdv{g_{2}}{u}\Delta u & \pdv{g_{2}}{v}\Delta v) = J_{\vb{g}} \Delta u\Delta v
\]
Si todo tiende a cero (infinitesimal) entonces
\begin{equation}
    \begin{split}
        \dd{x} \dd{y} = |J_{\vb{g}}| \dd{u} \dd{v}
    \end{split}
\end{equation}
\end{document}