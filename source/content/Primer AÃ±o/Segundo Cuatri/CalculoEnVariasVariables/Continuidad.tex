%! TEX root = Calculo.tex

\documentclass{./Calculo.tex}

\begin{document}
\chapter{Continuidad}
\begin{defin}
Dada $\mathbf{f}:D\subset \mathbb{R}^{n}\to \mathbb{R}^{m}$, si $\mathbf{a}$ es un punto límite de $D$, decimos que $\mathbf{f}(\mathbf{x})$ tiende a $\mathbf{b}$ cuando $\mathbf{x}$ tiende a $\mathbf{a}$, y se denota por
 
$$
\lim_{ \mathbf{x} \to \mathbf{a} } \mathbf{f}(\mathbf{x})=\mathbf{b}
$$
Si se cumple que
 
$$
\forall \varepsilon > 0 \exists\delta>0 : |\mathbf{x}-\mathbf{a}|<\delta \implies |\mathbf{f}(\mathbf{x}) - \mathbf{b}|<\varepsilon
$$

Esta definición es equivalente al límite
 
$$
\lim_{ |\mathbf{x}-\mathbf{a}| \to 0 } |\mathbf{f}(\mathbf{x})-\mathbf{b}| = 0 
$$

En estas definiciones no decimos cómo el punto $\mathbf{x}$ se acerca a $\mathbf{a}$. Por tanto, si el límite depende de por donde nos acerquemos a $\mathbf{a}$, entonces este no existe.
\end{defin}
Para evitar utilizar la definición a la hora de calcular límites, utilizaremos los siguientes teoremas.
\begin{teorema}
Sean $\mathbf{f}:D\to \mathbb{R}^{m}$ y $\mathbf{g}:C\to \mathbb{R}^{m}$ con $D\cap C \neq \phi$, y sea $\mathbf{a}$ un punto límite de $D\cap C$, y además suponemos que
$$
\lim_{ \mathbf{x} \to \mathbf{a} } \mathbf{f}(\mathbf{x})=\mathbf{b}
$$
$$
\lim_{ \mathbf{x} \to \mathbf{a} } \mathbf{g}(\mathbf{x})=\mathbf{c}
$$
Entonces se cumple que
- $\lim_{ \mathbf{x} \to \mathbf{a} } \mathbf{f}(\mathbf{x})+\mathbf{g}(\mathbf{x})=\mathbf{b}+\mathbf{c}$\\
- $\lim_{ \mathbf{x} \to \mathbf{a} } \lambda\mathbf{f}(\mathbf{x})=\lambda\mathbf{b}$\\
- $\lim_{ \mathbf{x} \to \mathbf{a} } \mathbf{f}(\mathbf{x})\cdot\mathbf{g}(\mathbf{x})=\mathbf{b}\cdot\mathbf{c}$\\
- $\lim_{ \mathbf{x} \to \mathbf{a} }|\mathbf{f}(\mathbf{x})|=|\mathbf{b}|$\\
\end{teorema}
\section{Continuidad}
\begin{defin}
Una función $\mathbf{f}$ es continua en un punto $\mathbf{a}$ si
$$
\lim_{ \mathbf{x} \to \mathbf{a} } \mathbf{f}(\mathbf{x})=\mathbf{f}(\mathbf{a})
$$
\end{defin}
\end{document}
