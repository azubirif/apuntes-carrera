\documentclass{amsbook}
\author{Alejandro Zubiri \& David Mateos}
\date{\today}
\title{Entrega: Homeomorfismos y conexidad}

\renewcommand*\contentsname{Índice}

\usepackage[margin=1.1in]{geometry}
\usepackage{amsmath, amsthm, amsfonts, mdframed, hyperref}

\begin{document}
\maketitle
\section{Ejercicio 1 (6 puntos)}
Demuestre que los únicos conjuntos conexos de \(\mathbb{R}\) son los intervalos, \(\mathbb{R}\) y el
conjunto vacío.\\
\subsection{Solución}
Solución aquí.
\section{Ejercicio 2 (3 puntos)}
Sea \(X\) un espacio topológico y sea \(Y = \{ 0,1 \}\) considerado como espacio
topológico con la topología discreta. Demuestre que si \(X\) es conexo, entonces
\(X\) no puede ser homeomorfo a \(Y\).\\
\subsection{Solución}
Bla bla bla
\end{document}