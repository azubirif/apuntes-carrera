\documentclass{amsbook}
\author{Alejandro Zubiri \& David Mateos}
\date{\today}
\title{Entrega: Homeomorfismos y conexidad}

\renewcommand*\contentsname{Índice}

\usepackage[margin=1.1in]{geometry}
\usepackage{amsmath, amsthm, amsfonts, mdframed, hyperref}

\begin{document}
\maketitle
\section{Ejercicio 1 (6 puntos)}
Demuestre que los únicos conjuntos conexos de \(\mathbb{R}\) son los intervalos, \(\mathbb{R}\) y el
conjunto vacío.\\
\subsection{Solución}
Empezamos definiendo ambos elementos. Un subconjunto $I$ es un intervalo
si para todo $x,y\in I, x < z < y, z \in I$. Por otro lado, un conjunto $E$
no es conexo si existen dos conjuntos abiertos $G,H$ tal que $G \neq \phi$,
$H \neq \phi$, $E = (E \cap G) \cup (E \cap H)$ y $G \cap H $.\\
Empezaremos demostrando que un conjunto es un intervalo sí y solo sí 
es conexo.
\textbf{Si es un intervalo, es conexo}: supongamos que existen dos conjuntos
$G,H$ que forman una desconexión en un intervalo $I$. Sea $a \in G \cap I$ y $b \in H \cap I$,
y asumamos que $a < b$. Puesto que $I$ es un intervalo, se tiene que $[a,b] \subseteq I$.
Ahora definamos
\section{Ejercicio 2 (3 puntos)}
Sea \(X\) un espacio topológico y sea \(Y = \{ 0,1 \}\) considerado como espacio
topológico con la topología discreta. Demuestre que si \(X\) es conexo, entonces
\(X\) no puede ser homeomorfo a \(Y\).\\
\subsection{Solución}
Si $X \cong Y$, existe una función continua y biyectiva $f: X \to Y$. Como
$Y$ tiene dos elementos, entonces hay dos casos:
\begin{itemize}
	\item Caso 1: $X$ tiene más o menos de dos elementos: entonces la función
		ya no es biyectiva.
	\item Caso 2: $X$ tiene dos elementos: entonces $X$ no es conexo.  
\end{itemize}
\end{document}
