\documentclass{amsbook}
\author{Alejandro Zubiri \& David Mateos}
\date{\today}
\title{Entrega: Homeomorfismos y conexidad}

\renewcommand*\contentsname{Índice}

\usepackage[margin=1.1in]{geometry}
\usepackage{amsmath, amsthm, amsfonts, mdframed, hyperref}

\begin{document}
\maketitle
\section{Ejercicio 1 (6 puntos)}
Demuestre que los únicos conjuntos conexos de \(\mathbb{R}\) son los intervalos, \(\mathbb{R}\) y el
conjunto vacío.\\
\subsection{Solución}
Empezamos definiendo ambos elementos. Un subconjunto $I$ es un intervalo
si para todo $x,y\in I, x < z < y, z \in I$. Por otro lado, un conjunto $E$
no es conexo si existen dos conjuntos abiertos $G,H$ tal que $G \neq \phi$,
$H \neq \phi$, $E = (E \cap G) \cup (E \cap H)$ y $G \cap H $.\\
Empezaremos demostrando que un conjunto es un intervalo sí y solo sí 
es conexo.\\
\textbf{Si es un intervalo, es conexo}: supongamos que existen dos conjuntos
$G,H$ abiertos que forman una desconexión en un intervalo $I$. Sea $a \in G \cap I$ y $b \in H \cap I$,
y asumamos que $a < b$. Puesto que $I$ es un intervalo, se tiene que $[a,b] \subseteq I$.
Ahora definamos $c= \sup((I \cap G)\cap [a,b])$. Con esta definición, tenemos 
que $c \geq a$ y que $c \leq b$, y puesto que $I$ es un intervalo, $c \in I$.\\
Si se cumple esto, hay dos opciones:
\begin{itemize}
  \item Si $c \in G \cap I$, como $G$ es abierto, $\exists \delta > 0 : (c-\delta, c + \delta) \subseteq G$.
    Pero entonces $c$ ya no sería el supremo, y por tanto $c$ no pertenece a $G \cap I$.
  \item Si $c \in H \cap I$, se tiene que $c \in H$. Como $H$ es abierto, tenemos que
    $\exists \delta > 0: (c -\delta, c) \subset H$. Como $a < c-\delta$, tenemos que 
    $(c-\delta, c) \subset I$, por lo que $(c-\delta, c)\subset H \cap I$, luego 
    $(c-\delta, c) \cap (G \cap I) = \phi$. Pero entonces $c < c-\delta$, que es una contradicción,
    luego $c$ no pertenece a $H \cap I$, luego $c$ no pertenece a $I$, que es una 
    contradicción, luego $I$ es conexo.
\end{itemize}
\textbf{Si es conexo, es un intervalo}: Sea $I$ un conjunto conexo que no es un intervalo,
entonces dados $a,b \in I, \exists c: a < c < b, c \notin I$. Sea entonces $U= (-\infty, c)$
y $V = (c, \infty)$. Tenemos que ambos son abiertos en $\mathbb{R}$, y que 
$a \in U$ y que $b \in V$, lo que implica que $I \cap U \neq \phi$ y que $I \cap V \neq \phi$.
Teniendo que son disjuntos, que $(I\cap U)\cup(I\cap V) = I$, y que luego
$(I\cap U)\cap (I \cap V) = \phi$, tenemos que $I$ no es conexo, lo que es una contradicción,
luego $I$ es un intervalo.\\
Habiendo demostrado que todos los intervalos son conexos, podemos definir $\mathbb{R} = (-\infty, \infty)$,
que es un intervalo, luego $\mathbb{R}$ es conexo, y $\phi = (a,a)$ para cualquier $a \in \mathbb{R}$,
luego $\mathbb{R}, \phi$ son conexos.
\section{Ejercicio 2 (3 puntos)}
Sea \(X\) un espacio topológico y sea \(Y = \{ 0,1 \}\) considerado como espacio
topológico con la topología discreta. Demuestre que si \(X\) es conexo, entonces
\(X\) no puede ser homeomorfo a \(Y\).\\
\subsection{Solución}
Si $X \cong Y$, existe una función continua y biyectiva $f: X \to Y$. Como
$Y$ tiene dos elementos, entonces hay dos casos:
\begin{itemize}
	\item Caso 1: $X$ tiene más o menos de dos elementos: entonces la función
		ya no es biyectiva.
	\item Caso 2: $X$ tiene dos elementos: entonces $X$ no es conexo.  
\end{itemize}
\end{document}
