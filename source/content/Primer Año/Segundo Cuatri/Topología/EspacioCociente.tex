\documentclass{./Topologia.tex}

\begin{document}
\chapter{Espacio Cociente}
\begin{defin}
    Sea \(X\) un espacio topológico, y sea \(Y\) un conjunto. Sea \(f:X \to Y\) una función
    sobreyectiva. La topología cociente determiada por \(f\) sobre \(Y\) de la siguiente forma.
    Un subconjunto \(A\) de \(Y\) es abierto si y solo si \(f^{-1}(A)\) es un subconjunto
    abierto de \(X\).
\end{defin}
Esta topología es la más grande sobre \(Y\).
\begin{prop}
    Un subconjunto \(C \subset Y\) es cerrado en \(Y\) si y solo si \(f^{-1}(C)\) es cerrado
    en \(X\).
\end{prop}
\begin{teorema}
    Sean \(X\) e \(Y\) espacios topológicos y \(\sim\) una relación de equivalencia sobre
    \(X\). Sea \(\phi :X / \sim \to Y\) una función y sea \(\pi: X \to  X / \sim\) la
    proyección canónica. Entonces \(\phi \) es continua si y solo si
    \[
        \phi \circ \pi: X \to Y
    \]
    es continua.
\end{teorema}
\begin{defin}
    Sean \(X\) e \(Y\) espacios topológicos y \(f:X \to  Y\) una aplicación. Se dice que
    \(f\) es cerrada si la imagen de cualquier conjunto cerrado de \(X\) es un conjunto cerrado
    de \(Y\).
\end{defin}
\end{document}