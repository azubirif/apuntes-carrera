%! TEX root = Geometria.tex

\documentclass{Geometria.tex}

\begin{document}
\chapter{Espacio proyectivo}
\section{Introducción}
Todas las rectas paralelas se cortan en el infinito. Supongamos que tenemos dos planos
\(\Pi_{1}\) y \(\Pi_{2}\), y una aplicación \(\phi \) que lleva puntos del primer plano al segundo
como la intersección de la recta que va de un punto del plano hasta un origen \(O\) y el segundo
plano.\\
Definimos el plano proyectivo de \(\mathbb{R}^{3}\) como
\[
    \mathbb{P}(\mathbb{R}^{3}) = \mathbb{R}^{2} - \{ 0 \} + P_{\infty}
\]
Tenemos que
\begin{itemize}
    \item Dos rectas siempre se cortan, aunque sea en el infinito.
\end{itemize}
Un punto en el plano proyectivo se define por sus coordenadas homogéneas, como un vector de una
recta. Se refleja en el plano afín con las coordenadas no homogéneas \((1, \frac{x_{1}}{x_{0}}
, \frac{x_{2}}{x_{0}})\) si \(x_{0} \neq 0\). Sino, como el vector \(0, x_{1}, x_{2}\).
\section{Espacio afín y proyectivo}
Si tenemos un espacio afín \(A\) de dimensión \(n\), entonces los puntos \(X\) se pueden ver como
puntos de \(\mathbb{P(R^{n+1})}\) como
\[
    A \to  \mathbb{P}(\mathbb{R}^{n+1})
\]
\[
    X \to (1: X)
\]
Los puntos que no son de la forma \((1:X)\) son puntos en el infinito o impropios.
\begin{defin}
    Se define una relación de equivalencia entre dos vectores si estos pertenecen a la misma recta.
    Se define el plano proyectivo como el conjunto cociente \(\frac{R^{3} - \{ 0 \}}{\sim}\).
\end{defin}
\begin{defin}
    Dado un espacio proyectivo \(\mathbb{P}(V)\), se llama subespacio proyectivo de dimensión \(k\)
    a un subconjunto de la forma \(\mathbb{P}(W)\) donde \(W\) es un subespacio vectorial
    de \(V\), de dimensión \(k+1\).
\end{defin}
\end{document}