%! TEX root = Geometria.tex

\documentclass{./Geometria.tex}

\begin{document}
\chapter{Espacio Afín Euclídeo}
Un espacio afín es euclídeo al incluir el \textbf{producto escalar}:
\begin{defin}
El producto escalar, que definimos como $<x,y> = x\cdot y$, nos permite medir distancias, entre otras cosas. Esto pasa a ser
\[
	\vb{x}\cdot \vb{y} = \sum x_{i}y_{i}
\]
Esta es una aplicación:
\[
	\cdot : \mathbb{R}^{n}\times \mathbb{R}^{n} \to \mathbb{R}
\]
Este cumple que:
\begin{itemize}
	\item Es simétrico.
	\item Es positivo siempre.
	\item Es una forma bilineal.
\end{itemize}
Figura \ref{fig:p-escalar}
\end{defin}
\begin{figure}[ht]
    \centering
    \incfig{p-escalar}
    \caption{Producto Escalar}
    \label{fig:p-escalar}
\end{figure}
Y ahora
\begin{defin}
Un espacio afín euclídeo es una cuaterna definida por $(A, \mathbb{V}, \phi, \cdot )$, formado por un espacio afín y el producto escalar.\\
La dimensión del EA euclídeo es la dimensión de su EA asociado.
\end{defin}
\section{Coordenadas covariantes y contravariantes}
\begin{defin}
	Dado un EA euclídeo sobre $\mathbb{R}$, con una base $\mathcal{B}= \{ \vb{v}_{i} \}$, llamamos \textbf{coordenadas contravariantes} del vector $\vb{u}\in E$, respecto a la base $\mathcal{B}$, a la $n$-upla $(a_1,\dots ,a_{n})$ de escalares tal que:
	\[
		\vb{u} = \sum^{n}_{i=1} a_{i} \vb{v}_{i}
	\]
\end{defin}
Por contraste, tenemos
\pagebreak
\begin{defin}
	Dado un EA euclídeo, las \textbf{coordenadas covariantes} de un vector $\vb{u}$, respecto a una base $\mathcal{B} = \{ \vb{v}_{i} \}$, son la $n$-upla definida por los escalares $(a^{*}_{1},\dots ,a^{*}_{n})$:
	\[
		a^{*}_{i} = \vb{v}_{i}\cdot \vb{u}
	\]
	Figura \ref{fig:coords-covariantes}\\
Lo importante de estas coordenadas, es que
\[
	<u_1,u_2><u_1^{*},u_2^{*}> = c
\]
Es decir, el producto de coordenadas covariantes entre coordenadas contravariantes es constante. Además, se cumplirá que si la base es ortonormal, ambas coincidirán.
\end{defin}
\begin{figure}[ht]
    \centering
    \incfig{coords-covariantes}
    \caption{Coordenadas Covariantes}
    \label{fig:coords-covariantes}
\end{figure}
Las coordenadas covariantes se pueden obtener multiplicando las coordenadas contravariantes por la \textbf{matriz de Gram}:
\[
	\begin{pmatrix} a_1^{*}\\\dots \\ a_{n}^{*} \end{pmatrix}=
	\begin{pmatrix} g_{11} & \dots & g_{1n}\\
		\dots & \dots &\dots \\
		g_{n 1} & \dots & g_{nn}
	\end{pmatrix} \begin{pmatrix} a_1\\ \dots \\ a_{n} \end{pmatrix} 
\]
\pagebreak
\section{Ángulos}
\begin{defin}
	Dado un EA euclídeo, y dos vectores $\vb{v}_{1}, \vb{v}_{2}$ no nulos, definimos el ángulo no orientado entre $\vb{v}_{1}$ y $\vb{v}_{2}$ como el $\arccos$ entre $0$ y $\pi$ de
	\[
		\cos \theta = \frac{\vb{v}_{1}\cdot \vb{v}_{2}}{|\vb{v}_{1}||\vb{v}_{2}|}
	\]
	Donde $|\vb{v}|$ es la norma del vector, definido como
	\[
		|\vb{v}| = \sqrt{\vb{v}\cdot \vb{v}}
	\]
\end{defin}
\begin{defin}[Perpendicularidad]
	Dado un EAE de dimensión $n$, y dos vectores $\vb{v}, \vb{u}$, estos dos son perpendiculares u ortogonales si y solo si
	\[
		\vb{v}\cdot \vb{u} = 0
	\]
	También, dos variedades afines $L_1=P_1+S_1$, $L_2=P_2+S_2$, estos serán perpendiculares si se cumple que:
	\begin{itemize}
		\item $L_1 \cap L_2 \neq \phi$
		\item $S_1 \perp S_2$ 
	\end{itemize}
\end{defin}
Ahora introducimos el siguiente teorema
\pagebreak
\begin{teorema}
	Dado un hiperplano $L$ cuya ecuación cartesiana respecto a un sistema rectangular (ortonormal), es
	\[
		\sum a_{i}x_{i} = c
	\]
	El vector de coordenadas $\vb{n} = (a_1,\dots ,a_{n})$ define rectas perpendiculares a $L$ y es el vector normal al hiperplano. Estas coordenadas son las \textbf{covariantes} del vector normal.  
\end{teorema}
\begin{defin}
	Dado un EAE de dimensión $n$, y un subespacio afín $L \subset A$ de dimensión $m$ y dirección $W$, entonces, para cada punto $P \in A$, existe un único subespacio afín $L'$ de $A$ por $P$, ortogonal a $L$ y con dimensión $n-m$. Este es
	\[
		L' = P + W^{\perp}
	\]
	La intersección entre ambas variadades, $L \cap L'$, es la proyección de $P$ sobre $L$.   
\end{defin}
\begin{figure}[ht]
    \centering
    \incfig{p-ortogonal}
    \caption{Proyección Ortogonal}
    \label{fig:p-ortogonal}
\end{figure}
\end{document}
