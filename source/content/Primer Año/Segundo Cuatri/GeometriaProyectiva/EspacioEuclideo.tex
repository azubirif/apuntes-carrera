%! TEX root = Geometria.tex

\documentclass{./Geometria.tex}

\begin{document}
\chapter{Espacio Afín Euclídeo}
Un espacio afín es euclídeo al incluir el \textbf{producto escalar}:
\begin{defin}
El producto escalar, que definimos como $<x,y> = x\cdot y$, nos permite medir distancias, entre otras cosas. Esto pasa a ser
\[
	\vb{x}\cdot \vb{y} = \sum x_{i}y_{i}
\]
Esta es una aplicación:
\[
	\cdot : \mathbb{R}^{n}\times \mathbb{R}^{n} \to \mathbb{R}
\]
Este cumple que:
\begin{itemize}
	\item Es simétrico.
	\item Es positivo siempre.
	\item Es una forma bilineal.
\end{itemize}
Figura \ref{fig:p-escalar}
\end{defin}
\begin{figure}[ht]
    \centering
    \incfig{p-escalar}
    \caption{Producto Escalar}
    \label{fig:p-escalar}
\end{figure}
Y ahora
\begin{defin}
Un espacio afín euclídeo es una cuaterna definida por $(A, \mathbb{V}, \phi, \cdot )$, formado por un espacio afín y el producto escalar.\\
La dimensión del EA euclídeo es la dimensión de su EA asociado.
\end{defin}
\section{Coordenadas covariantes y contravariantes}
\begin{defin}
	Dado un EA euclídeo sobre $\mathbb{R}$, con una base $\mathcal{B}= \{ \vb{v}_{i} \}$, llamamos \textbf{coordenadas contravariantes} del vector $\vb{u}\in E$, respecto a la base $\mathcal{B}$, a la $n$-upla $(a_1,\dots ,a_{n})$ de escalares tal que:
	\[
		\vb{u} = \sum^{n}_{i=1} a_{i} \vb{v}_{i}
	\]
\end{defin}
Por contraste, tenemos
\pagebreak
\begin{defin}
	Dado un EA euclídeo, las \textbf{coordenadas covariantes} de un vector $\vb{u}$, respecto a una base $\mathcal{B} = \{ \vb{v}_{i} \}$, son la $n$-upla definida por los escalares $(a^{*}_{1},\dots ,a^{*}_{n})$:
	\[
		a^{*}_{i} = \vb{v}_{i}\cdot \vb{u}
	\]
	Figura \ref{fig:coords-covariantes}\\
Lo importante de estas coordenadas, es que
\[
	<u_1,u_2><u_1^{*},u_2^{*}> = c
\]
Es decir, el producto de coordenadas covariantes entre coordenadas contravariantes es constante. Además, se cumplirá que si la base es ortonormal, ambas coincidirán.
\end{defin}
\begin{figure}[ht]
    \centering
    \incfig{coords-covariantes}
    \caption{Coordenadas Covariantes}
    \label{fig:coords-covariantes}
\end{figure}
Las coordenadas covariantes se pueden obtener multiplicando las coordenadas contravariantes por la \textbf{matriz de Gram}:
\[
	\begin{pmatrix} a_1^{*}\\\dots \\ a_{n}^{*} \end{pmatrix}=
	\begin{pmatrix} g_{11} & \dots & g_{1n}\\
		\dots & \dots &\dots \\
		g_{n 1} & \dots & g_{nn}
	\end{pmatrix} \begin{pmatrix} a_1\\ \dots \\ a_{n} \end{pmatrix} 
\]
\pagebreak
\section{Ángulos}
\begin{defin}
	Dado un EA euclídeo, y dos vectores $\vb{v}_{1}, \vb{v}_{2}$ no nulos, definimos el ángulo no orientado entre $\vb{v}_{1}$ y $\vb{v}_{2}$ como el $\arccos$ entre $0$ y $\pi$ de
	\[
		\cos \theta = \frac{\vb{v}_{1}\cdot \vb{v}_{2}}{|\vb{v}_{1}||\vb{v}_{2}|}
	\]
	Donde $|\vb{v}|$ es la norma del vector, definido como
	\[
		|\vb{v}| = \sqrt{\vb{v}\cdot \vb{v}}
	\]
\end{defin}
\begin{defin}[Perpendicularidad]
	Dado un EAE de dimensión $n$, y dos vectores $\vb{v}, \vb{u}$, estos dos son perpendiculares u ortogonales si y solo si
	\[
		\vb{v}\cdot \vb{u} = 0
	\]
	También, dos variedades afines $L_1=P_1+S_1$, $L_2=P_2+S_2$, estos serán perpendiculares si se cumple que:
	\begin{itemize}
		\item $L_1 \cap L_2 \neq \phi$
		\item $S_1 \perp S_2$ 
	\end{itemize}
\end{defin}
Ahora introducimos el siguiente teorema
\pagebreak
\begin{teorema}
	Dado un hiperplano $L$ cuya ecuación cartesiana respecto a un sistema rectangular (ortonormal), es
	\[
		\sum a_{i}x_{i} = c
	\]
	El vector de coordenadas $\vb{n} = (a_1,\dots ,a_{n})$ define rectas perpendiculares a $L$ y es el vector normal al hiperplano. Estas coordenadas son las \textbf{covariantes} del vector normal.  
\end{teorema}
\begin{defin}
	Dado un EAE de dimensión $n$, y un subespacio afín $L \subset A$ de dimensión $m$ y dirección $W$, entonces, para cada punto $P \in A$, existe un único subespacio afín $L'$ de $A$ por $P$, ortogonal a $L$ y con dimensión $n-m$. Este es
	\[
		L' = P + W^{\perp}
	\]
	La intersección entre ambas variadades, $L \cap L'$, es la proyección de $P$ sobre $L$.   
\end{defin}
\begin{figure}[ht]
    \centering
    \incfig{p-ortogonal}
    \caption{Proyección Ortogonal}
    \label{fig:p-ortogonal}
\end{figure}
\pagebreak
\section{Distancias}
\subsection{Distancia entre dos puntos}
\begin{defin}
Dado un EAE, y dos puntos $P,Q \in A$, la distancia entre los dos puntos es la norma del vector $\vb{PQ}$:
\[
	d(P,Q) = \norm{\vb{PQ}}
\]
Propiedades:
\begin{itemize}
	\item $d$ es definida positiva.
	\item $d$ es simétrica: $d(P,Q) = d(Q,P)$.
	\item $d$ cumple la desigualdad triangular: $d(P,Q) \leq d(P,R) + d(R,Q)$. 
\end{itemize}
\end{defin}
\subsection{Distancia entre variedades}
\begin{defin}
Dadas dos variedades $L_1=P_1+S_1$ y $L_2=P_2+S_2$, se define la distancia entre ellas como la menor distancia entre sus respectivos puntos:
\[
	d(L_1,L_2) = min(\{ d(X,Y) : X \in L_1, y \in L_2 \})
\]
Si se da que $L_1 \cap L_2 \neq \phi, d(L_1,L_2) = 0$.
\end{defin}
\subsection{Distancia entre punto y subespacio afín}
\begin{defin}
Dado un punto $P$ y un subespacio afín $L$, la distancia entre ambos es la distancia entre $P$ y la proyección ortogonal de $P$ y $L$:
\[
	d(P,L)= d(P,P_{L})
\]
\end{defin}
\section{Isometría}
\begin{defin}
Dados dos EAEs $(E, V, \phi)$ y $(E', V', \phi')$, diremos que una aplicación $f$ es una isometría si:
\begin{equation}
	\begin{split}
		f&: E \to E'\\
		(P,Q) &\to f(P,Q)
	\end{split}
\end{equation}
Y se debe cumplir que
\[
	\boxed{
		d'(f(P),f(Q)) = d(P,Q)
	}
\]
Donde $d$ es la distancia definida en $E$ y $d'$ es la distancia definida en $E'$.    
\end{defin}
\begin{center}
	

\end{center}
\begin{figure}[ht]
	\centering
	\caption{Isometrías}
	\label{fig:isometrias}


\tikzset{every picture/.style={line width=0.75pt}} %set default line width to 0.75pt

\begin{tikzpicture}[x=0.75pt,y=0.75pt,yscale=-1,xscale=1]
%uncomment if require: \path (0,300); %set diagram left start at 0, and has height of 300

%Shape: Ellipse [id:dp16206097342092396]
\draw   (90.2,157.19) .. controls (90.2,108.52) and (113.22,69.06) .. (141.61,69.06) .. controls (170,69.06) and (193.01,108.52) .. (193.01,157.19) .. controls (193.01,205.86) and (170,245.32) .. (141.61,245.32) .. controls (113.22,245.32) and (90.2,205.86) .. (90.2,157.19) -- cycle ;
%Shape: Ellipse [id:dp3924000237283274]
\draw   (405.99,152.78) .. controls (405.99,104.11) and (429,64.66) .. (457.39,64.66) .. controls (485.78,64.66) and (508.8,104.11) .. (508.8,152.78) .. controls (508.8,201.45) and (485.78,240.91) .. (457.39,240.91) .. controls (429,240.91) and (405.99,201.45) .. (405.99,152.78) -- cycle ;
%Straight Lines [id:da7556011380605056]
\draw    (163.34,117.53) -- (428.66,113.16) ;
\draw [shift={(430.66,113.13)}, rotate = 179.06] [color={rgb, 255:red, 0; green, 0; blue, 0 }  ][line width=0.75]    (10.93,-3.29) .. controls (6.95,-1.4) and (3.31,-0.3) .. (0,0) .. controls (3.31,0.3) and (6.95,1.4) .. (10.93,3.29)   ;
%Straight Lines [id:da5411924729180644]
\draw    (154.53,196.85) -- (419.85,196.85) ;
\draw [shift={(421.85,196.85)}, rotate = 180] [color={rgb, 255:red, 0; green, 0; blue, 0 }  ][line width=0.75]    (10.93,-3.29) .. controls (6.95,-1.4) and (3.31,-0.3) .. (0,0) .. controls (3.31,0.3) and (6.95,1.4) .. (10.93,3.29)   ;
%Straight Lines [id:da9279665366462678]
\draw    (142.8,121.85) -- (137.99,172.86) ;
\draw [shift={(137.8,174.85)}, rotate = 275.39] [color={rgb, 255:red, 0; green, 0; blue, 0 }  ][line width=0.75]    (10.93,-3.29) .. controls (6.95,-1.4) and (3.31,-0.3) .. (0,0) .. controls (3.31,0.3) and (6.95,1.4) .. (10.93,3.29)   ;
%Straight Lines [id:da36632798481514883]
\draw    (459.89,126.28) -- (457.88,177.85) ;
\draw [shift={(457.8,179.85)}, rotate = 272.24] [color={rgb, 255:red, 0; green, 0; blue, 0 }  ][line width=0.75]    (10.93,-3.29) .. controls (6.95,-1.4) and (3.31,-0.3) .. (0,0) .. controls (3.31,0.3) and (6.95,1.4) .. (10.93,3.29)   ;

% Text Node
\draw (133.86,44.09) node [anchor=north west][inner sep=0.75pt]    {$E$};
% Text Node
\draw (451.03,40.62) node [anchor=north west][inner sep=0.75pt]    {$E'$};
% Text Node
\draw (138.54,107.22) node [anchor=north west][inner sep=0.75pt]    {$P$};
% Text Node
\draw (130.44,180.65) node [anchor=north west][inner sep=0.75pt]    {$Q$};
% Text Node
\draw (441.39,104.28) node [anchor=north west][inner sep=0.75pt]    {$f( P)$};
% Text Node
\draw (437.69,182.12) node [anchor=north west][inner sep=0.75pt]    {$f( Q)$};


\end{tikzpicture}

\end{figure}
\textbf{Propiedades}:
\begin{itemize}
	\item La composición de isometrías una isometría.
	\item Las isometrías afines consevan los ángulos entre subespacios afines:
		\[
			\cos (\vb{u}, \vb{v}) = \cos (\bar{f}(\vb{u}), \bar{f}(\vb{v}))
		\]
\end{itemize}
\section{Ejemplos de transformaciones}
\begin{defin}
Una traslación por el vector $\vb{v}$ es una transformación afín que lleva un punto $P$ al punto $P + \vb{v}$.
\[
	P \mapsto t(P) = P+\vb{v}
\]
\end{defin}
\begin{defin}[Proyección ortogonal]
La aplicación
\[
	\delta:A \mapsto A
\]
\[
	P \mapsto \delta(P)
\]
donde $\delta(P)$ es el único punto del conjunto $L \cap L'$, y es una proyección. $\delta(P)$ es la proyección sobre $L$ con dirección $W^{\perp}$, y su aplicación lineal asociada es la proyección vectorial ortogonal sobre $W$.
\end{defin}
\pagebreak
\begin{defin}[Homotecia]
Sea $(E, V, \phi)$ un EAE, y sea $\lambda \in K - \{ 0,1 \}$, y sea un punto $O \in A$. Una homotecia es una aplicación $h: E \to E$ que deja $O$ fijo, y que tiene por aplicación lineal asociada una homotecia vertical de razón $\lambda$:
\begin{equation}
	\begin{split}
		h&: E \mapsto E\\
		P &\mapsto O + \lambda \vb{OP}
	\end{split}
\end{equation}
$h$ es una homotecia de razón $\lambda$ y centro $O$.\\
Si es una homotecia, la aplicación afín asociada de la homotecia es:
\[
	\bar{h} = \lambda Id
\]
Donde $\lambda$ es la razón de la homotecia.\\
Propiedades:
\begin{itemize}
	\item La homotecia es biyectiva.
	\item La composición de homotecias es una homotecia.
	\item El conjunto de homotecias de centro $C$, junto con la identidad, es un grupo.
\end{itemize}
\end{defin}
Para calcular el centro de la homotecia, teniendo la razón $\lambda$, y la definición de $h(P) = O + \lambda \vb{OP}$, el centro es
\[
	O = P- \frac{1}{1-\lambda} \vb{P f(P)}
\]
\end{document}
