%! TEX root = Geometria.tex

\documentclass{./Geometria.tex}

\begin{document}
\chapter{Variedades afines y posiciones relativas}
\section{Variedades Afines}
Se define como un subespacio afín de dimensión finita, y estos se pueden definir de diferentes formas.
\[
	S= \{ P_0, \dots , P_{n} \}
\]
Este conjunto de puntos generan el subespacio vectorial asociado:
\[
	W=\{ \mathbf{P_0P_1},\dots ,\mathbf{P_0P_{n}} \}
\]
Y estos generan el subespacio afín
\[
	L=P+W
\]
También, dado un conjunto de vectores $\{ \mathbf{u}_{i} \}$  que formen un sistema generador, junto con un punto, podríamos generar la variedad.\\
Todas estas definiciones conforman la misma variedad afín.
\begin{defin}

	Sea $A$ un espacio afín $(A, V, \phi)$ asociado a un espacio vectorial $V$. Sea $L$ un subespacio afín de $A$, y $P \in L$, y $U$ un subespacio vectorial de $V$.\\
	Se llama variedad lineal de $A$ que pasa por $P$ y con dirección $U$ al conjunto de puntos
	\begin{equation}
		\begin{split}
			P+U=\{ P+\mathbf{u}: \mathbf{u} \in U \}
		\end{split}
	\end{equation}
También se da que en un espacio afín $(A,V,\phi)$, $A$ es variedad lineal de $A$.   
\end{defin}
Dada una variedad afín $P+W$ y un punto $Q\in P+W$, entonces
\[
	P+W=Q+W
\]
lo que quiere decir que podemos definir la misma variedad afín con cualquier punto que pertenezca a esta.\\
Dado un subconjunto de puntos $L$ de $A$, podemos comprobar si es una variedad afín si:
\begin{itemize}
	\item Que el conjunto de vectores generado por los puntos pertenece a $V$.
	\item en tal caso, comprobar que $L=P+\{ \mathbf{AB}, A, B \in L \}$ para algún punto de $L$.  
\end{itemize}
\begin{defin}

	Dado un espacio afín $A$ de dimensión $n$ y $L=P+W$ una variedad lineal de $A$ con dirección $W$. Dado que $L$ es espacio afín sobre $W$, entonces
	\[
		dim(L)=dim(W)
	\]

\end{defin}
El conjunto de soluciones de un sistema lineal
\[
	PX=C
\] 
de $m$ ecuaciones con $n$ incógnitas, compatible, es una variedad lineal de $\mathbb{R} ^{n}$.
\subsection{Ecuaciones de una variedad lineal}
Sea $L=P+W$ una variedad lineal de $\mathbb{R}^{n}$ de dimensión $n$, tal que $W$ está generado por una base de vectores $\{ \mathbf{v}_{1},\dots ,\mathbf{v}_{k} \}$. Para cualquier punto $X \in L$, se cumple que
\[
	\mathbf{PX}= \sum_{i=1}^{k} \lambda_{i} \mathbf{v}_{i}
\]
Ahora, siendo $O$ el origen, entonces un punto $X \in L$ se puede expresar como
\[
	\mathbf{OX} = \mathbf{OP} + \sum_{i=1}^{k} \lambda_{i} \mathbf{v}_{i}
\]
que llamaremos \textbf{ecuación vectorial}.
\subsection{Ecuaciones paramétricas de una VA}
Si conocemos las coordenadas de un punto $P$, podemos sustituir en la ecuación vectorial, y tras igualar coordenadas obtenemos las ecuaciones paramétricas:
\begin{equation*}
	\begin{split}
		x_1 &= p_1 + \lambda_1v_{11}+\dots +\lambda_{k}v_{k_1}\\
		\dots \\
		x_{n} &= p_{n} + \lambda_{n}v_{1n} + \dots +\lambda_{k}v_{kn}
	\end{split}
\end{equation*}
\subsection{Ecuaciones cartesianas}
Si despejamos los valores $\lambda_{i}$, obtenemos las ecuaciones cartesianas.
\section{Variedades afines en el plano $A_{2}$ }
Teniendo el sistema de referencia
\[
	R = \{ O, \mathbf{e}_{1}, \mathbf{e}_{2} \}
\]
que define un punto por sus coordenadas
\[
	P=(p_1,p_2)
\]
\subsection{Rectas}
Una recta es un \textbf{hiperplano}, determinada por
\begin{itemize}
	\item Un punto $P(p_1,p_2)$
	\item Un vector de dirección $\mathbf{v}=(v_1,v_2) \neq \mathbf{0}$ 
\end{itemize}
Este vector de dirección determina el subespacio de dirección:
\[
	S=L(\mathbf{v})
\]
Si una recta viene determinada por dos puntos $P,Q$, podemos expresarla de forma continua como
\[
	\frac{x-p_1}{q_1-p_1}= \frac{y-p_2}{q_2-p_2}
\]
Siendo el vector director de esta el vector $\mathbf{PQ}$. En el caso particular en el que $P=(a, 0)$ y $Q=(0,b)$, se da que
\[
	\frac{x}{a}+\frac{y}{b}=1
\]
\section{Variedades afines en $A_{3}$ }
Teniendo un sistema de referencia
\[
	R = \{ O, \mathbf{e}_{1}, \mathbf{e}_{2}, \mathbf{e}_{3} \}
\]
Un punto se define por sus coordenadas
\[
	P=(p_1,p_2,p_3)
\]
\subsection{Rectas}
Las rectas se definen (de nuevo) por un punto $P$ y un vector director $\mathbf{u}$, y cualquier punto de la recta se define como
\[
	X = P +\lambda \mathbf{u}
\]
Si una recta viene determinada por dos puntos $P,Q$, esta se puede expresar como
\[
	\frac{x-p_1}{q_1-p_1} = \frac{y-p_2}{q_2-p_2}= \frac{z-p_3}{q_3-p_3}
\]
Que es su forma de ecuación continua, con vector director $\mathbf{PQ}$.\\
\subsubsection{Tipos de ecuaciones de una recta}
\begin{itemize}
	\item Implícitas: expresadas como la intersección de dos planos.
	\item Vectorial: punto más vector.
	\item Continua: despejar los parámetros.
	\item Paramétrica: una ecuación para cada componente de un punto.
\end{itemize}
\subsection{Planos}
Un plano se define como una base $\{ \mathbf{u}, \mathbf{v} \}$ apoyado sobre un punto $P$. La base define todos los puntos que se pueden expresar de la forma
\[
	X = P + \lambda \mathbf{u} + \mu \mathbf{v}
\]
que serán los puntos que pertenecen al plano. También podemos definir un plano con tres puntos $P,Q,R$, siendo la base los vectores $\mathbf{PQ}$ y $\mathbf{PR}$, apoyados sobre cualquiera de los tres puntos.
\begin{equation}
	\begin{split}	
	\pi \equiv \begin{vmatrix}
		x-p_1 & u_1 & v_1 \\
		y - p_2 & u_2 & v_2 \\
		z - p_3 & u_3 & v_3
	\end{vmatrix} = 0
	\end{split}
\end{equation}
\section{Haces de variedades afines}
Se llama \textbf{haz de rectas} al conjunto de rectas del plano que pasan por un punto, este se denomina \textbf{vértice}. La ecuación del haz de rectas que pasa por el punto $P$ es:
\[
	\alpha(x-x_0)+\beta(y-y_0)=0
\]
\subsection{Haces de rectas dado por dos rectas}
Dadas dos rectas $r,s$ por sus ecuaciones cartesianas, el haz de rectas formado por $r$ y $s$ es el conjunto de todas las rectas que pasan por su interseccion.
\subsection{Haces de planos en $A^{3}$ }
Dados dos planos $\pi_1, \pi_2$, definimos el haz de planos como todos los planos que contienen a la intersección de los planos $\pi_1$ y $\pi_2$.\\
Para calcular este haz, como necesitamos tres puntos, tomamos dos puntos que pertenezcan a la intersección de los planos, y dejamos el resto de valores como parámetros. Geométricamente, el plano puede "girar" alrededor de la recta. También, podemos escribir sumar la ecuación de $\pi_1$ más $k\cdot \pi_2$ e igualarlo a $0$.\\
Por otro lado, para obtener un haz de planos \textbf{paralelos}, simplemente obtenemos la normal del plano, y dejamos la constante final $D$ como parámetro.
\subsection{Recta afín que se apoya en dos rectas}
Una recta que se apoya en otras dos es aquella que corta de forma \textbf{perpendicular} a ambas rectas, además de que pasa por un punto concreto. Ambas rectas deben cruzarse y no ser paralelas.
\section{Posiciones relativas}
\subsection{Posiciones relativas de dos variedades afines}
Dadas dos variedades lineales afines $L_1=P+W_1$ y $L_2=Q+W_2$, con sus respectivas ecuaciones, entonces la intersección de estas variedades son las soluciones de las ecuaciones que representan ambas variedades. Analizando la matriz de los coeficientes de las ecuaciones, tenemos que
\begin{itemize}
	\item Las variedades se cortan si la matriz es compatible, es decir, el rango de la matriz de coeficientes $A$ es igual al rango de la matriz ampliada $A'$.
	\item Las variedades se cruzan ambos rangos no coinciden.
	\item Son paralelas si no hay intersección.
\end{itemize}
\end{document}
