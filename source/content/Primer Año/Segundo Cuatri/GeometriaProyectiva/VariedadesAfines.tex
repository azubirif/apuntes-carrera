\documentclass{article}
\author{Alejandro Zubiri}
\title{Variedades Afines}

\renewcommand*\contentsname{Índice}

\usepackage[margin=1.1in]{geometry}
\usepackage{amsmath, physics, amsthm, amsfonts, mdframed, subfiles, tikz}
\usepackage[a]{esvect}

\newmdtheoremenv{teorema}{Teorema}
\newmdtheoremenv{defin}{Definición}

\newcommand{\R}{\mathbb{R}}

\begin{document}
\maketitle
\tableofcontents
\pagebreak
\section{Variedades Afines}
Se define como un subespacio afín de dimensión finita, y estos se pueden definir de diferentes formas.
\[
	S= \{ P_0, \dots , P_{n} \}
\]
Este conjunto de puntos generan el subespacio vectorial asociado:
\[
	W=\{ \mathbf{P_0P_1},\dots ,\mathbf{P_0P_{n}} \}
\]
Y estos generan el subespacio afín
\[
	L=P+W
\]
También, dado un conjunto de vectores $\{ \mathbf{u}_{i} \}$  que formen un sistema generador, junto con un punto, podríamos generar la variedad.\\
Todas estas definiciones conforman la misma variedad afín.
\begin{defin}

	Sea $A$ un espacio afín $(A, V, \phi)$ asociado a un espacio vectorial $V$. Sea $L$ un subespacio afín de $A$, y $P \in L$, y $U$ un subespacio vectorial de $V$.\\
	Se llama variedad lineal de $A$ que pasa por $P$ y con dirección $U$ al conjunto de puntos
	\begin{equation}
		\begin{split}
			P+U=\{ P+\mathbf{u}: \mathbf{u} \in U \}
		\end{split}
	\end{equation}
También se da que en un espacio afín $(A,V,\phi)$, $A$ es variedad lineal de $A$.   
\end{defin}
Dada una variedad afín $P+W$ y un punto $Q\in P+W$, entonces
\[
	P+W=Q+W
\]
lo que quiere decir que podemos definir la misma variedad afín con cualquier punto que pertenezca a esta.\\
Dado un subconjunto de puntos $L$ de $A$, podemos comprobar si es una variedad afín si:
\begin{itemize}
	\item Que el conjunto de vectores generado por los puntos pertenece a $V$.
	\item en tal caso, comprobar que $L=P+\{ \mathbf{AB}, A, B \in L \}$ para algún punto de $L$.  
\end{itemize}
\begin{defin}

	Dado un espacio afín $A$ de dimensión $n$ y $L=P+W$ una variedad lineal de $A$ con dirección $W$. Dado que $L$ es espacio afín sobre $W$, entonces
	\[
		dim(L)=dim(W)
	\]

\end{defin}
El conjunto de soluciones de un sistema lineal
\[
	PX=C
\] 
de $m$ ecuaciones con $n$ incógnitas, compatible, es una variedad lineal de $\mathbb{R} ^{n}$.
\subsection{Ecuaciones de una variedad lineal}
Sea $L=P+W$ una variedad lineal de $\mathbb{R}^{n}$ de dimensión $n$, tal que $W$ está generado por una base de vectores $\{ \mathbf{v}_{1},\dots ,\mathbf{v}_{k} \}$. Para cualquier punto $X \in L$, se cumple que
\[
	\mathbf{PX}= \sum_{i=1}^{k} \lambda_{i} \mathbf{v}_{i}
\]
Ahora, siendo $O$ el origen, entonces un punto $X \in L$ se puede expresar como
\[
	\mathbf{OX} = \mathbf{OP} + \sum_{i=1}^{k} \lambda_{i} \mathbf{v}_{i}
\]
que llamaremos \textbf{ecuación vectorial}.
\subsection{Ecuaciones paramétricas de una VA}
Si conocemos las coordenadas de un punto $P$, podemos sustituir en la ecuación vectorial, y tras igualar coordenadas obtenemos las ecuaciones paramétricas:
\begin{equation*}
	\begin{split}
		x_1 &= p_1 + \lambda_1v_{11}+\dots +\lambda_{k}v_{k_1}\\
		\dots \\
		x_{n} &= p_{n} + \lambda_{n}v_{1n} + \dots +\lambda_{k}v_{kn}
	\end{split}
\end{equation*}
\subsection{Ecuaciones cartesianas}
Si despejamos los valores $\lambda_{i}$, obtenemos las ecuaciones cartesianas. 
\end{document}
