%! TEX root = Programacion.tex

\documentclass{./Programacion.tex}

\begin{document}
\chapter{Python}
\section{Programación Orientada a Objetos (POO)}
Permite programar realizando abstracciones. Consiste en definir plantillas que desarrollan como cada \textbf{objeto} (instancias de las clases) definirá sus atributos y los diferentes métodos que estos tienen.
\begin{itemize}
	\item Clase:
		\begin{itemize}
			\item Atributos
			\item Métodos
		\end{itemize}
\end{itemize}
\subsection{Encapsulamiento}
Es el concepto de aislar el funcionamiento de los métodos y variables de una clase.
\section{Declaración de clases}
\begin{verbatim}
	class Clase:
		pass

	objeto = Clase()
\end{verbatim}
Cada clase tiene atributos predefinido, pero luego tenemos los atributos de \textbf{instancia}, que permite al usuario modificar los atributos en su creación:
\begin{verbatim}
	class Clase:
	def __init__(self, at1, at2):
		self.var1 = at1
		self.var2 = at2

	obj = Clase(1, 2)
	obj.var1 #1
	obj.var2 #2
\end{verbatim}
Las diferencias entre ambos es que los atributos de clase se almacenan en los metadatos de la clase, mientras que los de instancia se guardan en los metadatos del objeto.
\subsection{Métodos}
Son funciones definidas dentro de una propia clase, y representan las acciones que cada instancia de dicha clase puede realizar. Estos métodos también pueden acceder a los atributos de una clase, por lo que es una forma de encapsular estos.
Existen dos tipos de métodos:
\begin{itemize}
	\item Métodos de instancia: relacionados con cada objeto.
	\item Métodos de clase: solo se pueden usar con la clase.
\end{itemize}
\end{document}
