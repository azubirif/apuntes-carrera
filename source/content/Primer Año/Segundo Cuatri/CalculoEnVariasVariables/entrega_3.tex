\documentclass[12pt,a4paper,reqno]{article}


\usepackage{amssymb}
\usepackage{enumerate}
\usepackage{enumitem}
\setlist{topsep=1.5em, itemsep=1.5em}
\usepackage[exercisename=Problema,printsolution=true,solutionname=Solución]{exercises}
\usepackage{hyperref, physics}
\usepackage{graphicx}
\usepackage{multicol}
\usepackage{wrapfig2}
\usepackage{bm}

\usepackage{tikz}
\usetikzlibrary{calc,intersections,through,backgrounds}
\usetikzlibrary {arrows.meta}

\newcommand{\bbR}{\mathbb{R}}
\newcommand{\cC}{\mathcal{C}}
\newcommand{\dparcial}[2]{\frac{\partial#1}{\partial#2}}
\newcommand{\derivbda}[2]{\frac{d#1}{d#2}}
\newcommand{\dom}{\text{dom }}
\newcommand{\exterior}{\text{ext }}
\newcommand{\evbluar}[2]{\left.#1\right|_{#2}}
\newcommand{\frontera}{\partial}
\newcommand{\interior}{\text{int }}
\newcommand{\sgn}{\text{sgn}}
\newcommand{\union}{\cup}

%\newtheoremstyle{plain}
\newtheorem{theorem}{Teorema}
\newtheorem{corolary}{Corolario}
\newtheorem{definition}{Definición}
\newtheorem{lemma}{Lema}

%\newtheoremstyle{definition}
\newtheorem{ejemplo}{Ejemplo}
\newtheorem{ejercicios}{Ejercicios}

\let\oldemptyset\emptyset
\let\emptyset\vbrnothing


\title{Derivbdas de orden superior, desarrollos de Taylor y regla de la cadena}
\date{\today}

\author{Alejandro Zubiri}
%Si sois vbrias personas entonces quitad el símbolo % de las siguientes lineas y ponédselo a la anterior
%\author{Nombre1 Apellidos1\\
	%	Nombre2 Apellidos2\\
	%	Nombre3 Apellidos3\\
	%	Nombre4 Apellidos4}



\begin{document}
	\maketitle
	\section{Problema 1}
	Hallar los desarrollos de Taylor de orden 2 en torno a los puntos que se indican.
	\begin{enumerate}[label={(\alph*)}]
		\item \textbf{[2 puntos]} $f(x,y)=(x-y)^2$ en$(1,2)$.
		
		\subsection*{Solución}
		Empezamos evaluando y desarrollando las derivadas:
		\begin{itemize}
			\item $f(1,2)=(1-2) ^{2}=1$
			\item $\pdv{f}{x}(1,2)=2-4=-2$
			\item $\pdv{f}{y}(1,2)=4-2=2$
			\item $\grad{f}=(-2,2)$ 
			\item $\pdv{f}{x}{y}(1,2)=-2$ 
			\item $\pdv{f}{y}{x}(1,2)=-2$
			\item $\pdv[2]{f}{x}(1,2)=2$
			\item $\pdv[2]{f}{y}(2,2)=2$ 
			\item $\grad{f}(1,2)\cdot (\mathbf{x}-\mathbf{a})=-2x+2y-2$ 
		\end{itemize}
	También, obtenemos que la correspondiente matriz Hessiana es
	\[
	H_{f}(\mathbf{a})= \begin{pmatrix}
		2 & -2 \\ -2 & 2
	\end{pmatrix}
	\]
	Ahora ya podemos desarrollar el polinomio de Taylor:
	\begin{equation}
		\begin{split}
			P_{2,(1,2)}f(\mathbf{x})&=1-2x+2y-2+ \frac{1}{2}(x-1,y-2) \begin{pmatrix}
				2 & -2 \\ -2 & 2	
			\end{pmatrix} \begin{pmatrix}
				x-1 \\ y -2
			\end{pmatrix}\\
			&=1 -2x +2y -2 +x^{2}+2x-2xy+y^{2}+1-2y\\
			&=x ^{2}+y ^{2} -2xy
		\end{split}
	\end{equation}
	Que es el polinomio de Taylor buscado		
\item \textbf{[2 puntos]} $g(x,y)=(1+x^2+y^2)^{-1}$ en $(0,0)$.
		
		\subsection*{Solución}
		\begin{itemize}
			\item $\pdv{g}{x}(x,y) = \frac{-2x}{(1+x ^{2}+y ^{2}) ^{2}}$
			\item $\pdv{g}{y}(x,y)= \frac{-2y}{(1+x ^{2}+y ^{2})^{2}}$
			\item $\pdv{g}{x}{x}(x,y)= \frac{8x ^{2}}{(1+x ^{2}+y ^{2})^{3}}$
			\item $\pdv{g}{y}{y}(x,y)= \frac{8y ^{2}}{(1+x ^{2}+y ^{2})^{3}}$
			\item $\pdv{g}{x}{y}(x,y)= \frac{8xy}{(1+x ^{2}+y ^{2})^{3}}$
			\item $\pdv{g}{y}{x}(x,y)= \frac{8xy}{(1+x ^{2}+y ^{2})^{3}}$ 
		\end{itemize}
		Si evaluamos cada una de estas derivadas en $(0,0)$, vemos claramente que tanto el gradiente como la correspondiente matrix Hessiana tienen componentes que son $0$, y por tanto solo sobrevive el término $g(x,y)=1$\footnote{Si observamos la función, vemos que esta solo se hace más pequeña a medida que $x$ o $y$ crecen, por lo que $g(0,0)$ es, de hecho, un máximo.}.   
		\[
			P_{2,(0,0)}(g(\mathbf{x}))=1
		\]
		\item \textbf{[2 puntos]} $h(x,y)=e^{xy}\cos(x+y)$ en $(0,\pi)$.
		
		\subsection*{Solución}
		\begin{itemize}
			\item $h(0, \pi)=1$
			\item $\grad{h}=e^{xy}(y\cos(x+y)-\sin(x+y),x\cos(x+y)-\sin(x+y))$
			\item $\grad{h}(0,\pi)=(-\pi,0)$
			\item $\grad{h}\cdot (\vb{x}-\vb{a})=-\pi x$
			\item $H_{h}(0,\pi)= \begin{pmatrix}
					-\pi ^{2}+1 & 0 \\ 0 & 1
			\end{pmatrix}
			$
		\item $H_{h}(\vb{a})\cdot (\vb{x}-\vb{a})= \begin{pmatrix} 
				x-x\pi ^{2}\\y-\pi
			\end{pmatrix} $ 
		\item $(\vb{x}-\vb{a})^{T}\cdot H_{h}(\vb{a})\cdot (\vb{x}-\vb{a})=
				x ^{2}-x ^{2}\pi ^{2}+y ^{2}+ \pi ^{2} -2y\pi
			$ 
		\end{itemize}		
	Y ahora, el polinomio es:
	\begin{equation}
		\begin{split}
			P_{2,(0,\pi)} = -1-\pi x + \frac{x ^{2}}{2}(1-\pi ^{2}) +\frac{1}{2} y ^{2} +\frac{1}{2} \pi ^{2} -y \pi
		\end{split}
	\end{equation}
	\end{enumerate}
	
	\section{Problema 2}
	Sean $\bm f(u,v)=\left(e^{u+2v},2u+v\right)$ y $\bm g(x,y,z)=\left(2x^2-y+3z^3,2y-x^2\right)$. Calcular la diferencial de $\bm f\circ\bm g$ en el punto $\bm a=(2,-1,1)$, de las siguientes maneras
	\begin{enumerate}[label={(\alph*)}]
		\item \textbf{[1 punto]} utilizando la regla de la cadena,
		
		\subsection*{Solución}
		Tenemos que
		\[
			D(f \circ g)=D f(g)Dg
		\]
		Y ahora
		\[
			D f(\vb{x})= \begin{pmatrix} 
				e^{u+2v} & 2e^{u+2v} \\2 & 1
			\end{pmatrix} 
		\]
		Sustituyendo $\vb{x}=\vb{g}(x,y,z)$, tenemos que
		\[
			D f(g)= \begin{pmatrix} 
				e^{3y+3z ^{3}} & 2 e^{3y +3z ^{3}} \\ 2 & 1
			\end{pmatrix} 
		\]
		Ahora pasamos a calcular $Dg$:
		\[
			Dg = \begin{pmatrix} 
				4x & -1 & 9z ^{3} \\ -2x & 2 & 0
			\end{pmatrix} 
		\]
		Ahora multiplicamos ambas para obtener el diferencial deseado:
		\[
			\vb{f} \circ \vb{g} = D(f(g))Dg = \begin{pmatrix} 
				0 & 3e^{3(y+z ^{3})} & 9z ^{2}e^{3(y+z^{3})}\\
				6x & 0 & 18z ^{2}
			\end{pmatrix} 
		\]
		\item \textbf{[1 punto]} componiendo y diferenciando.
		
		\subsection*{Solución}
		Componemos ambas funciones para obtener
		\[
			\vb{f}\circ \vb{g} = (e^{3y+3z^{3}},3x ^{2}+6z ^{3})
		\]
		y ahora tomamos el diferencial de esta función:
		\[
			D(\vb{f}\circ\vb{g}) = \begin{pmatrix} 
				0 & 3e^{3(y+z^{3})} & 9z^{2}e^{3(y+z^{3})}\\
				6x & 0 & 18z^{2}
			\end{pmatrix} 
		\]
		Que como podemos observar, coinciden.
	\end{enumerate}
	
	\section{Problema 3 [2 puntos]}
	Las ecuaciones $u=f(x,y,z)$, $x=s^2+t^2$, $y=s^2-t^2$, $z=2st$ definen $u$ en función de $s$ y $t$: $u=F(s,t)$. Expresar las derivbdas segundas de $F$ respecto a $s$ y $t$ en función de las derivbdas de $f$ ($f\in\cC^2$).
	
	\subsection*{Solución}
	Tenemos que
	\[
		u = f(x,y,z)=f(s^{2} + t ^{2}, s ^{2}-t ^{2}, 2st)=F(s,t)
	\]
	Empezamos con las derivadas primeras:
	\begin{equation}
		\begin{split}
			\pdv{F}{s}&= \pdv{f}{x}\pdv{x}{s} + \pdv{f}{y}\pdv{y}{s} +\pdv{f}{z}\pdv{z}{s}\\
					  &= 2s \pdv{f}{x}+2s\pdv{f}{y}+2t\pdv{f}{z}
		\end{split}
	\end{equation}
	A partir de ahora se reducirán algunos cálculos para hacer el resultado más ameno. Los cálculos completos se pueden encontrar en el otro documento (donde están las soluciones hechas a mano).
	\begin{equation}
		\begin{split}
			\pdv{F}{t} = 2t\pdv{f}{x}-2t\pdv{f}{y}+2s\pdv{f}{z}
		\end{split}
	\end{equation}
	Y ahora pasamos a las derivadas segundas:
	\begin{equation}
		\begin{split}
			\pdv[2]{F}{s} &= 2\pdv{f}{x} +4s ^{2}\pdv[2]{f}{x}-4s^{2}\pdv{f}{x}{y} +4st \pdv{f}{x}{z}\\
						  &+2\pdv{f}{y} - 4s^{2}\pdv{f}{y}{x}+4s^{2}\pdv[2]{f}{y}-4st\pdv{f}{y}{z}\\
						  &+4st\pdv{f}{z}{x}-4st\pdv{f}{z}{y}+4t^{2}\pdv[2]{f}{z}
		\end{split}
	\end{equation}
	\begin{equation}
		\begin{split}
			\pdv[2]{F}{t} &= 2\pdv{f}{x}+4t^{2}\pdv[2]{f}{x}-4t^{2}\pdv{f}{x}{y}-4st\pdv{f}{x}{z}\\
						  &-2\pdv{f}{y}-4t^{2}\pdv{f}{y}{x}+4t^{2}\pdv[2]{f}{y}-4st\pdv{f}{y}{z}\\
						  &+4st\pdv{f}{z}{x}-4st\pdv{f}{z}{y}+4s^{2}\pdv[2]{f}{z}
		\end{split}
	\end{equation}
	\begin{equation}
		\begin{split}
			\pdv{F}{s}{t} &= 4st\pdv[2]{f}{x}-4st\pdv{f}{x}{y}+4s^{2}\pdv{f}{x}{z}\\
						  &+4st\pdv{f}{y}{x}-4st\pdv[2]{f}{y}+4s^{2}\pdv{f}{y}{z}\\
						  &+2\pdv{f}{z}+4t^{2}\pdv{f}{z}{x}-4t^{2}\pdv{f}{z}{y}+4st\pdv[2]{f}{z}
		\end{split}
	\end{equation}
	\begin{equation}
		\begin{split}
			\pdv{F}{t}{s} &= 4st\pdv[2]{f}{x}+4st\pdv{f}{x}{y}+4t^{2}\pdv{f}{x}{z}\\
						  &-4st\pdv{f}{y}{x}-4st\pdv[2]{f}{y}-4t^{2}\pdv{f}{y}{z}\\
						  &+2\pdv{f}{z}+4s^{2}\pdv{f}{z}{x}+4s^{2}\pdv{f}{z}{y}+4st\pdv[2]{f}{z}
		\end{split}
	\end{equation}
	\end{document}
