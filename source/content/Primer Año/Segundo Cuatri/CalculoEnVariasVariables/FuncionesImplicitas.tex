%TEX root = Calculo.tex

\documentclass{../Calculo.tex}

\begin{document}
\chapter{Funciones Implícitas}
\section{Funciones Inversas}
Dada $f: A \to B$, se dice que $f$ tiene inversa si existe una función $g: B \to A$ tal que
\[
	f\circ g = Id_{b}
\]
y similarmente
\[
	g \circ f = Id_{a}
\]
Si esta función existe, se denota por
\[
	f^{-1} \neq \frac{1}{f}
\]
Se puede demostrar que $f^{-1}$ existe si y solo si $f$ es inyectiva y sobreyectiva (biyectiva). Sin embargo, dadas ciertas funciones que no sean ni inyectivas ni sobreyectivas, se pueden restringir de forma que sean biyectivas.\\
\textbf{Ejemplos}:
\begin{itemize}
	\item $f(x)=x ^{2}$. Esta función es sobreyectiva, pero no es inyectiva, y por tanto no tiene inversa. Sin embargo, podemos restringir el dominio tal que $x\geq 0$, y entonces sí que es inyectiva, por tanto biyectiva, y por tanto tiene inversa.
	\item $g(x)=\cos x$. Esta función es periódica, pero si restringimos la función a un período $[0, \pi]$, entonces es biyectiva.  
\end{itemize}
Sea $\bar{f}(\mathbf{x}) = \mathbf{y}$. Queremos ver si podemos invertir esta relación $(\bar{f}^{-1})$, y obtener
\[
	\mathbf{x} = \bar{f}^{-1}(\mathbf{y})
\]
Para ello, introducimos el siguiente teorema.
\begin{teorema}[Teorema de la función inversa]
	Dada una función $f:\mathbb{R}^{n} \to \R^{n}$, esta tiene inversa si y solo si el determinante jacobiano (determinante de su diferencial) es distinto a $0$.  
\end{teorema}

\end{document}
