%! TEX root = Calculo.tex

\documentclass{./Calculo.tex}

\begin{document}
\chapter{Integrales múltiples}
\section{Integrales dobles}
\begin{defin}
    Los borelianos en \(\mathbb{R}^{2}\) son la familia mínima que contiene
    a los conjuntos
    \[
        A \times B
    \]
    con \(A,B \in \mathcal{B}(\mathbb{R})\) y que cumple:
    \begin{itemize}
        \item ser cerrada bajo complementación.
        \item ser cerrada bajo uniones numerables.
    \end{itemize}
    A esta familia la denotaremos por \(\mathcal{B}(\mathbb{R}^{2})\) o
    \(\mathcal{B}(\mathbb{R}) \otimes \mathcal{B}(\mathbb{R})\).
\end{defin}
También necesitaremos una medida para estos borelianos, por lo que usaremos la
\textbf{medida producto}. Dado
\[
    A \times B \in  \mathcal{B}(\mathbb{R}^{2})
\]
entonces
\[
    \mu_{2}(A \times B) = \mu(A)\cdot \mu(B)
\]
Observemos que si, por ejemplo, \(A 0 \{ a \} \subset \mathbb{R}\) entonces
\[
    \mu_{2}(\{ a \} \times B) = 0
\]
Los conjuntos unidimensionales tienen medida \(\mu_{2}\) nula.
\end{document}