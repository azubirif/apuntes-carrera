%! TEX root = Probabilidad.tex

\documentclass{./Probabilidad.tex}

\begin{document}
\chapter{Variable Continua}
Definida por variables que toman valores reales.
\begin{defin}
	Definimos la función densidad $\rho: \mathbb{R} \to \mathbb{R}$ como la que da la probabilidad de que $x$ se halle en un intervalo $[a,b]$.
	\begin{itemize}
		\item $\rho(x)> 0$.
		\item $\int_{-\infty}^{\infty} \rho(x)\dd{x} = 1$ 
	\end{itemize}
	Sin embargo, ya no podemos encontrar la probabilidad de que $x$ tome un valor específico. Solo podemos calcular:
	\begin{equation}
		\begin{split}
			P(a \leq x \leq b) = \int_{a}^{b} \rho(x) \dd{x}
		\end{split}
	\end{equation}
\end{defin}
Ahora, definimos
\begin{equation}
	\begin{split}
		E[x] = \int x \rho(x) \dd{x}
	\end{split}
\end{equation}
ahora nuestra función distribución se define como
\begin{equation}
	\begin{split}
		F(x_0) = P(x \leq x_0) = \int_{-\infty}^{x_0} \rho(x) \dd{x}
	\end{split}
\end{equation}
Por tanto, tenemos que si $F(x)$ es derivable en $x_0$, entonces
\begin{equation}
	\begin{split}
		\rho(x_0) = \dv{F}{x}(x_0)
	\end{split}
\end{equation}
\textbf{Propiedades}:
\begin{itemize}
	\item $P(a,b]= F(b) - F(a)$ 
	\item $P[a,b] = F(b) - F(a^{-})$ 
	\item $P(a) = F(a^{+}) - F(a^{-})$ 
\end{itemize}
\section{Percentiles}
Sea $p$ un número definido entre 0 y 1. El percentil $p$ de una distribución se denota como $\eta(100p)$ y está definido como:
\begin{equation}
	\begin{split}
		p = F(\eta(100p)) = \int_{-\infty}^{x_{M}}\rho(t)\dd{t}
	\end{split}
\end{equation}
En este caso, resolveríamos la ecuación de igualar nuestra función distribución a $\frac{1}{2}$ si quisiéramos obtener la mediana. 
\section{Probabilidad condicionada}
Dada una variable continua \(X\) con función densidad \(f(x)\), se define la función densidad de la
variable aleatoria continua condicionada \(X | A\) como
\[
	f(X | A) = \frac{f(x)}{P(A)}
\]
\section{Distribución exponencial}
Se define como
\[
	f(x) =  \left\{\begin{matrix}
		\lambda e^{-\lambda x}~x \geq 0 \\ 0~ x < 0
		\end{matrix}\right.
\]
y se da que la esperanza es
\[
	E[X] = \frac{1}{\lambda }
\]
y su función distribución
\[
	F(x) = 1 -e^{-\lambda x}
\]
además de \(V(x) = \frac{1}{\lambda^{2}}\) y \(med = \frac{\ln 2}{\lambda }\).
\\
Cuando queremos ver la probabilidad de que $x$ esté a una distancia de
$k$ desviaciones de la media, buscamos
\[
	P (-k \leq \frac{x - \mu}{\sigma} \leq k)
\]
\\
Cuando $n$ tiende a $30$, la distribución binomial se aproxima a una normal.  
\section{Desigualdad de Chebyshev}
Supongamos una variable aleatoria $X$ con media $\mu$ y desviación típica
$\sigma$. Se verifica que
\[
	P(|x - \mu| >k \sigma) \leq \frac{1}{k ^{2}} 
\]
\end{document}
