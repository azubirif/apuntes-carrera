%! TEX root = Topologia.tex

\documentclass{./Topologia.tex}

\begin{document}
\chapter{Espacios Métricos}
\begin{defin}
Sea un conjunto $X$. Una función $d: X \times X \to \mathbb{R}$ es una métrica o una función distancia sobre $X$ si $d$ cumple que:
\begin{itemize}
	\item $d(x,y) > 0 \forall x,y \in X$. $d(x,y)=0 \iff x=y$.
	\item $d(x,y) = d(y,x)$
	\item $\forall x,y,z \in X, d(x,z) \leq d(x,y) + d(y,z)$ 
\end{itemize}
Al par $(X,d)$ se le llama espacio métrico. 
\end{defin}
\begin{defin} 
	Sea $V$ un espacio vectorial real, y sean $x, y, z \in V$. Decimos que el punto $z$ está entre $x$ e $y$ si y solo si existe $t \in \mathbb{R}$, $0 \leq t \leq 1$ tal que $z = tx +(1-t)y$.     

\end{defin}
\begin{prop}
Sea $(V, \cdot )$ un espacio vectorial con producto interno sobre $\mathbb{R}$. La igualdad en la desigualdad triangular se cumple si y solo si $z$ está entre $x$ e $y$.     
\end{prop}
\begin{prop}
	Sea $x \in \mathbb{R}^{n}$. La norma euclidiana se define como:
	\[
		\norm{x} = \sqrt{\sum x_{i}^{2}}
	\]
	La norma euclidiana es una norma.
\end{prop}
\begin{defin}
Definimos el diámetro de un conjunto y la distancia de un punto a un conjunto como:
\[
	diam(A) = \sup \{ d(x,y): x,y \in A \}
\]
y la distancia de un punto a un conjunto:
\[
	d(x,A) = \inf \{ d(x,y) : y \in A \}
\]
\end{defin}
\begin{defin}
	Sea $(X,d)$ un espacio métrico. Sea $x \in X$ y $r > 0$. Los subconjuntos
	\begin{align}
		B_{d}(x,r) = \{ y \in X : d(x,y) < r \}
	\end{align}
	Se denominan \textbf{bolas abiertas} con centro en $x$ y radio $r$.
\end{defin}
Se da que en $\mathbb{R}$, los intervalos abiertos son bolas abiertas.
\begin{defin}
	Sea $V$ un espacio vectorial real, y $x,y \in  V$. Sea
	\begin{align}
		[x,y] = \{ (1-t)x+ty: 0 \leq t \leq 1 \}
	\end{align}
	A este conjunto lo llamaremos \textbf{segmento de recta}. Un subconjunto
	$A$ de $V$ es convexo si para cualquier par de elementos $x$ e $y$ en
	el segmento $[x,y] \subset A$.
\end{defin}
\begin{teorema}
	Sea $(X,d)$ un espacio métrico y un punto $x$. La bola abierta
	$B(x,r)$ es un abierto en $X$.
\end{teorema}
\begin{defin}
	Decimos que un subconjunto $J \subset \mathbb{R}$ es un intervalo
	si y solo si para todo $x,y \in J$ y para todo $z: x < z < y$ o
	al revés, se cumple que $z \in J$.
\end{defin}
Sea \(X,d\) un espacio métrico. Sea \(\mathcal{T}\) el conjunto de todos los subconjuntos
abiertos de \(X\). Entonces \(\mathcal{T}\) satisface que:

\begin{itemize}
	\item \(\phi, X \in  \mathcal{T}\)
	\item La unión arbitraria de elementos de \(\mathcal{T}\) pertenece a \(\mathcal{T}\).
	\item La intersección de dos elementos de \(\mathcal{T}\) está en \(\mathcal{T}\).
\end{itemize}
A \(\mathcal{T}\) se le llama la topología determinada por \(d\).
\end{document}
